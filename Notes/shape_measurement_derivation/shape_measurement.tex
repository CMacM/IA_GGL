\documentclass[onecolumn,amsmath,aps,fleqn, superscriptaddress]{revtex4}
 
\usepackage{amssymb}
\usepackage{amsmath}
\usepackage{epsfig}
\usepackage{subfigure}
\usepackage{mathrsfs}
\usepackage{longtable}
\usepackage{enumerate} 
\usepackage{multirow}
\usepackage{color}

\usepackage[usenames,dvipsnames]{xcolor}
\usepackage{hyperref}
\hypersetup{
    colorlinks = true,
    citecolor = {MidnightBlue},
    linkcolor = {BrickRed},
    urlcolor = {BrickRed}
}

\newcommand{\be}{\begin{eqnarray}}
\newcommand{\ee}{\end{eqnarray}}

\allowdisplaybreaks[1]
 
\begin{document}

\title{Intrinsic alignments from different shape measurement techniques}

\author{Danielle Leonard}

\maketitle

\paragraph*{Background} Different shape measurement methods are sensitive to different parts of the radial profile of a galaxy. In Singh et al. 2016, it appears as though the intrinsic alignment signals measured by these different methods are offset from one another by a constant. By measuring a correlation between galaxy position and shape with two different shape measurements, and subtracting these, you could get rid of the lensing signal and obtain a constraint on the scale-dependence of intrinsic alignments. This is useful for building intrinsic alignment models.

\paragraph*{Source sample} This method naturally allows us to loosen the standard assumption that only excess galaxies are subject to IA. We consider galaxies which form part of the smooth source distribution, but which are still physically close to a lens galaxy, to be impacted by intrinsic alignments as well. We assume a source sample which is defined by the requirement that the photo-z of a source galaxy is always within a certain cut-off separation of the lens galaxy in question. In the case of an effective redshift for the lenses, call the high and low photo-z edges of the sample $z_{\rm cut_h}$ and $z_{\rm cut_l}$, respectively. Note that galaxies with photo-z below the effective lens redshift are included in the sample, because they are subject to intrinsic alignments. 

We write:
\begin{align}
\tilde{\gamma}(r_p) &= \frac{\sum\limits_{k}^{\rm lens} \tilde{w}_k \tilde{\gamma}_k}{\sum\limits_{k}^{\rm lens} \tilde{w}_k} = \frac{\sum\limits_{k}^{\rm lens} \tilde{w}_k \gamma_{\rm L}^k}{\sum\limits_{k}^{\rm lens} \tilde{w}_k}+\frac{\sum\limits_{k}^{\rm lens} \tilde{w}_k \gamma_{\rm IA}^k}{\sum\limits_{k}^{\rm lens} \tilde{w}_k} \nonumber \\ 
\tilde{\gamma}'(r_p) &= \frac{\sum\limits_{k}^{\rm lens} \tilde{w}_k \tilde{\gamma}_k}{\sum\limits_{k}^{\rm lens} \tilde{w}_k} = \frac{\sum\limits_{k}^{\rm lens} \tilde{w}_k \gamma_{\rm L}^k}{\sum\limits_{k}^{\rm lens} \tilde{w}_k}+\frac{\sum\limits_{k}^{\rm lens} \tilde{w}_k a \gamma_{\rm IA}^k}{\sum\limits_{k}^{\rm lens} \tilde{w}_k} 
\label{full_both}
\end{align}
where all shear should be assumed to be tangential about the lens galaxy. When computing averaged tangential shear about a lens galaxy, the inverse-variance weights are:
\begin{equation}
\tilde{w}_k = \frac{1}{e_{\rm rms}^2 + (\sigma_e^k)^2}
\label{weights}
\end{equation}

A couple of notes on equation \ref{full_both}:
\begin{enumerate}
\item{The optimal choice for the two-point function here is the averaged tangential shear rather than $\Delta \Sigma$. $\Delta \Sigma$ is ill-defined for source galaxies which are very near in redshift to the lens galaxy or in front of the lens galaxy. By definition $\Delta \Sigma$ will be ill-defined for much of our sample.}
\item{The weights in equation \ref{full_both} are the same for both shape measurement methods. This is not an obvious thing to choose, because $e_{\rm rms}$ would be expected to vary between the methods. But, it's important to choose the weights the same so that lensing will cancel when we do the subtraction. Also, it turns out that because $e_{\rm rms}$ dominates over measurement error, choosing a non-optimal weight by changing $e_{\rm rms}$ has little effect on how optimal the weights are, because we are basically weighting every pair the same regardless. So, for the weights, we pick some $e_{\rm rms}$ which is between the values for the two shape measurement methods.}
\item{The boost factor is not included in equation \ref{full_both}. This is because the point of the boost factor is that if you are trying to get a measurement of lensing by summing tangential shears, then you need the boost factor or else your measured signal will be low. But here, we are not trying to get the lensing signal. We are interested in the intrinsic alignment signal, and the lensing signal will cancel with or without the boost, so it is not necessary.}
\end{enumerate}

We subtract the above relationships for the two shape measurement methods to get:
\begin{align}
\tilde{\gamma}(r_p) - \tilde{\gamma}^\prime(r_p) &= \frac{\sum\limits_{k}^{\rm lens} \tilde{w}_k \gamma_{\rm IA}^k}{\sum\limits_{k}^{\rm lens} \tilde{w}_k} - \frac{\sum\limits_{k}^{\rm lens} \tilde{w}_k a \gamma_{\rm IA}^k}{\sum\limits_{k}^{\rm lens} \tilde{w}_k} 
\label{subtract_gammat}
\end{align}
In principle, all the galaxies in our source sample contribute to intrinsic alignments, so we can consider an average per-galaxy contribution of $\gamma_{\rm IA}(r_p)$.
\begin{equation}
\tilde{\gamma}(r_p) - \tilde{\gamma}^\prime(r_p) = \gamma_{\rm IA}(r_p)(1-a).
\label{subtract_gammat_simp}
\end{equation}
However, there is another factor here that we have not yet accounted for. Because we are assuming photometric redshifts for our sources, we will have source galaxies in our sample which are actually higher redshift than $z_{\rm cut_h}$ and at lower redshift than $z_{\rm cut_l}$. These galaxies will not be affected by intrinsic alignments. Therefore, if we apply equation \ref{subtract_gammat_simp} directly, we will estimate the intrinsic alignment signal to be too low. We need to apply a correction factor which accounts for this. 

\section*{Correction factor: inclusion of higher-z and lower-z galaxies}
We assume that we know the smooth distribution of galaxies, $dN / dz$, from which our source sample is drawn to some reasonably high redshift, from a spectroscopic subsample. This may be difficult for future surveys, in which an inadequate spectroscopic subsample could result in systematic error (discussed below). 

Let $z_{\rm min}$ and $z_{\rm max}$ be, respectively, the minimum and maximum spec-z, and let the probability of labelling a galaxy with spectroscopic redshift $z_s$ as having photometric redshift $z_p$ be $p(z_s, z_p)$. The required correction factor is the fraction of galaxies in the photo-z-defined sample which have spec-z between $z_{\rm cut_l}$ and $z_{\rm cut_h}$. This can be separated into two terms:
\begin{enumerate}
\item{The galaxies from the smooth redshift distribution which have spec-z between $z_{\rm cut_l}$ and $z_{\rm cut_h}$.}
\item{The excess galaxies. All of these have spec-z between $z_{\rm cut_l}$ and $z_{\rm cut_h}$ by definition.}
\end{enumerate}
Schematically, we have:
\begin{equation}
N_{\rm corr} = \frac{{\rm smooth, z_s \, in \, (z_{\rm cut_l}, z_{\rm cut_h})} + {\rm excess}}{{\rm total}} =  \frac{{\rm smooth, z_s \, in \, (z_{\rm cut_l}, z_{\rm cut_h})}}{{\rm total}} + \frac{{\rm excess}}{{\rm total}}
\label{schematic}
\end{equation}
where these quantities should all be considered as weighted. {\it Should they?}

The weighted fraction of galaxies which are excess, i.e., the second term above, is given by:
\begin{equation}
1 - \frac{1}{B(r_p)}.
\label{frac_excess}
\end{equation}

The numerator of the first term on the right hand side of equation \ref{schematic}, that is, the weighted number of galaxies from the smooth distribution which have spec-z within the photo-z cuts, is given by:
\begin{equation}
\int_{z_{\rm cut_l}}^{z_{\rm cut_h}} dz_p \tilde{w}(z_p) \int_{z_{\rm cut_l}}^{z_{\rm cut_h}} dz_s\frac{dN}{dz_s} p(z_s, z_p)
\label{num_smooth_speczin}
\end{equation}

The total weighted number of galaxies from the {\it smooth} distribution can be given similarly by:
\begin{equation}
\int_{z_{\rm cut_l}}^{z_{\rm cut_h}} dz_p  \tilde{w}(z_p) \int_{z_{\rm min}}^{z_{\rm max}} dz_s\frac{dN}{dz_s} p(z_s, z_p)
\label{tot_smooth}
\end{equation}
However, this is not the denominator of the first term on the right hand side of equation \ref{schematic}, for this we need the total weighted number of galaxies, not the total only from the smooth distribution. Fortunate, the boost is defined as (schematically):
\begin{equation}
B(r_p) = \frac{ {\rm total}}{{\rm total, smooth}}
\label{boost_schm}
\end{equation}
It can therefore be seen that multiplying equation \ref{tot_smooth} by the boost gives the desired total. Assembling these terms, we find the correction factor to be given by:
\begin{equation}
N_{\rm corr}(r_p) = \frac{\int_{z_{\rm cut_l}}^{z_{\rm cut_h}} dz_p \tilde{w}(z_p) \int_{z_{\rm cut_l}}^{z_{\rm cut_h}} dz_s\frac{dN}{dz_s} p(z_s, z_p)}{B(r_p) \int_{z_{\rm cut_l}}^{z_{\rm cut_h}} dz_p  \tilde{w}(z_p) \int_{z_{\rm min}}^{z_{\rm max}} dz_s\frac{dN}{dz_s} p(z_s, z_p)} +1 - \frac{1}{B(r_p)}.
\label{Ncorr}
\end{equation}
If desired, this can be rearranged into the less pedagogical but potentially more useful form:
\begin{equation}
N_{\rm corr}(r_p) = 1 - \frac{1}{B(r_p)}\left(1 - \frac{\int_{z_{\rm cut_l}}^{z_{\rm cut_h}} dz_p \tilde{w}(z_p) \int_{z_{\rm cut_l}}^{z_{\rm cut_h}} dz_s\frac{dN}{dz_s} p(z_s, z_p)}{\int_{z_{\rm cut_l}}^{z_{\rm cut_h}} dz_p  \tilde{w}(z_p) \int_{z_{\rm min}}^{z_{\rm max}} dz_s\frac{dN}{dz_s} p(z_s, z_p)} \right)
\label{ncorr_2}
\end{equation}

We alter equation \ref{subtract_gammat_simp} to account for this correction factor
\begin{equation}
\tilde{\gamma}(r_p) - \tilde{\gamma}^\prime(r_p) = \gamma_{\rm IA}(r_p)(1-a) N_{\rm corr}(r_p)
\label{subtract_gammat_fac}
\end{equation}
which is trivially rearranged to solve for the IA term:
\begin{equation}
\gamma_{\rm IA}(r_p)(1-a) = \left(\tilde{\gamma}(r_p) - \tilde{\gamma}^\prime(r_p)\right)\left(\frac{1}{N_{\rm corr}(r_p)}\right)
\label{IAsol}
\end{equation}

\section*{Statistical error}

There are two types of statistical error sources in equation \ref{IAsol}: 
\begin{enumerate}
\item{$\tilde{\gamma}(r_p)$ and $\tilde{\gamma}^\prime(r_p)$ have statistical error due to shape noise}
\item{$N_{\rm corr}(r_p)$ has statistical error due to $B(r_p)$, which is sourced from large scale structure}
\end{enumerate}
It is expected that the second of these will be sub-dominant, but we include it anyway for completeness. 

These two sources of error are independent, so the total statistical error contribution can be written as:
\begin{align}
\sigma^2(\gamma_{\rm IA}(r_p)(1-a)) &= (1-a_{\rm fid})^2\gamma_{\rm IA}^{\rm fid}(r_p)^2\left(\frac{\sigma^2(N_{\rm corr}(r_p))}{N_{\rm corr}^{\rm fid}(r_p)^2} + \frac{\sigma^2(\tilde{\gamma}(r_p) - \tilde{\gamma}^\prime(r_p))}{(\tilde{\gamma}(r_p) - \tilde{\gamma}^\prime(r_p))^2}\right) \nonumber \\
&= (1-a_{\rm fid})^2\gamma_{\rm IA}^{\rm fid}(r_p)^2\left(\frac{\sigma^2(N_{\rm corr}(r_p))}{N_{\rm corr}^{\rm fid}(r_p)^2} + \frac{\sigma^2(\tilde{\gamma}(r_p) - \tilde{\gamma}^\prime(r_p))}{(N_{\rm corr}^{\rm fid}(1-a_{\rm fid})\gamma_{\rm IA}^{\rm fid}(r_p))^2}\right)
\label{error}
\end{align}
Note that in the above equation we explicitly consider only diagonal elements of the covariance matrix between $r_p$ bins. {\it The statistical error contribution from $N_{\rm corr}$ may have off-diagonal elements. For the moment we don't consider these.}

$\sigma^2(N_{\rm corr}(r_p))$ is given by:
\begin{equation}
\sigma^2(N_{\rm corr}(r_p)) = \frac{\sigma^2(B(r_p))}{B(r_p)^4}\left[1-\frac{\int_{z_{\rm cut_l}}^{z_{\rm cut_h}} dz_p \tilde{w}(z_p) \int_{z_{\rm cut_l}}^{z_{\rm cut_h}} dz_s\frac{dN}{dz_s} p(z_s, z_p)}{\int_{z_{\rm cut_l}}^{z_{\rm cut_h}} dz_p  \tilde{w}(z_p) \int_{z_{\rm min}}^{z_{\rm max}} dz_s\frac{dN}{dz_s} p(z_s, z_p)}\right]
\label{Ncorr_err}
\end{equation}

The error on the averaged tangential shear obtained using each shape-measurement method is dominated by shape noise, so the covariance can be represented by a diagonal matrix:
\begin{equation}
\sigma^2(\gamma(r_p)) = \frac{e_{\rm rms}^2}{N_{ls}(r_p)}
\label{diag_corr}
\end{equation} 
where $e_{\rm rms}$ is different for each shape measurement (as opposed to the average $e_{\rm rms}$ which is used in computing the weights). The errors for the two shape measurements are correlated:
\begin{equation}
\sigma^2(\tilde{\gamma}(r_p) - \tilde{\gamma}^\prime(r_p)) = \sigma^2(\tilde{\gamma}(r_p)) + \sigma^2(\tilde{\gamma}^\prime(r_p)) - 2 {\rm Cov}( \tilde{\gamma}(r_p), \, \tilde{\gamma}^\prime(r_p))
\label{comb_gammat}
\end{equation}
where ${\rm Cov}( \tilde{\gamma}(r_p), \, \tilde{\gamma}^\prime(r_p))$ is calculated as:
\begin{equation}
{\rm Cov}( \tilde{\gamma}(r_p), \, \tilde{\gamma}^\prime(r_p)) = f \left(\sqrt{\sigma^2(\tilde{\gamma}(r_p), \tilde{\gamma}(r_p))}\right)\left(\sqrt{\sigma^2(\tilde{\gamma}^\prime(r_p), \tilde{\gamma}^\prime(r_p))}\right)
\label{Cov}
\end{equation}
where $f$ is the fractional correlation between the two shape measurement methods (empirical input).

\paragraph{Note on computing $N_{ls}$}: {\it THIS SHOULD BE FIXED UP IN PRINCIPLE.} We have the surface density of galaxies for the full survey at all redshifts. To compute $N_{ls}$ from theory, we require the fraction of these which lie within the photo-z cuts of our source sample. This includes both galaxies from the smooth source sample and excess galaxies. This is given by:
\begin{equation}
\frac{B^{\rm samp}(r_p) \int_{z_{\rm cut_l}}^{z_{\rm cut_h}} dz_p \tilde{w}(z_p) \int_{z_{\rm min}}^{z_{\rm max}} dz_s \frac{dN}{dz_s}p(z_s, z_p)}{B^{\rm tot}(r_p) \int_{z_{\rm min}}^{z_{\rm max}} dz_p \tilde{w}(z_p) \int_{z_{\rm min}}^{z_{\rm max}} dz_s \frac{dN}{dz_s}p(z_s, z_p)}
\label{Nls}
\end{equation}
where there are two boost factors, one for our photo-z-defined sample and one for the full survey. We expect that the boost factor for the full survey will be smaller than that for our sample, because of the addition of comparatively more smooth background galaxies than excess galaxies. We assume that both boost factors follow the same power law in $r_p$: $(B(r_p)-1) \propto r^{-0.8}$, with a free parameter of the value of $B(r_p = 1 {\rm Mpc/h})$. To set this for each boost, we look to Blazek et al. 2012. There in Figure 1 the boost is measured for various samples. For now, we associate the `assoc' sample with our source sample, and the `src' sample with the total survey, {\it although the latter is not strictly correct as `src' includes no sources at $z$ below the lens}. Therefore, from Figure 1 of Blazek et al. 2012, we fiducially set $B^{\rm samp}(r_p = 1 {\rm Mpc/h}) = 1.2$ and $B^{\rm tot}(r_p = 1 {\rm Mpc/h}) = 1.04$.

Also to note is the fact that equation \ref{Nls} gives the weighted fraction of galaxies in our source sample. In the current place, where weights are assumed to be independent of $z$, this is equivalent to the actual fraction. {\it However, in the case in which weights were redshift-dependent, this could affect the result.}

\section*{Systematic error}

There are similarly two obvious sources of systematic error in equation \ref{IAsol}:
\begin{enumerate}
\item{$\frac{dN}{dz_s}$: We may have an imperfect knowledge of $\frac{dN}{dz_s}$ due to an inadequate spectroscopic subsample, especially for dim source galaxies.}
\item{$p(z_s, z_p)$: We will not necessarily have an accurate model of how likely we are to measure $z_p$ given $z_s$.}
\end{enumerate}

Uncertainty in either $\frac{dN}{dz_s}$ or $p(z_s, z_l)$ leads to uncertainty in the $\frac{1}{N_{\rm corr}}$ term of equation \ref{IAsol}. This manifests as a systematic error here because it affects the measurements of the IA term in each projected $r_p$ bin in the same way. For example, if $\frac{1}{N_{\rm corr}}$ is high, then $(1-a)\gamma_{\rm IA}(r_p)$ would be high in each bin. This introduces a correlation between $r_p$ bins as well as a contribution to the diagonal elements to the covariance. {\it For the moment, we are only considering the diagonal elements of the covariance.}

The systematic error contributions from these sources are given by:
\begin{align}
S_{dN}^2 &= \sigma_{dN}^2 (1-a_{\rm fid})^2 N_{\rm corr}^{{\rm fid}}(r_p) N_{\rm corr}^{{\rm fid}}(r_p')\gamma_{\rm IA}^{\rm fid}(r_p) \gamma_{\rm IA}^{\rm fid}(r_p') \nonumber \\
S_{p}^2 &= \sigma_{p}^2 (1-a_{\rm fid})^2 N_{\rm corr}^{{\rm fid}}(r_p) N_{\rm corr}^{{\rm fid}}(r_p')\gamma_{\rm IA}^{\rm fid}(r_p) \gamma_{\rm IA}^{\rm fid}(r_p') 
\label{S2}
\end{align}
where `fid' indicates a fiducial value.

In order to get an estimate for $\sigma_{\rm dN}$, we would vary $\frac{dN}{dz}$. This could be done, for example, by varying the parameters of $\frac{dN}{dz}$ over the range of their uncertainty (given in Nakajima 2011 for SDSS). Then, for each of these choices of $\frac{dN}{dz}$, we would compute $\frac{1}{N_{\rm corr}}$. The range of resultant $N_{\rm corr}$ would allow for an estimate of $\sigma_{\rm dN}$. A similar scheme could be used to account for $\sigma_{\rm p}$.

\subsection{Fiducial $\gamma^{\rm IA}(r_p)$}
To calculate both the systematic and statistical error, we need a fiducial model for $\gamma^{\rm IA}(r_p)$.

\subsubsection*{Excess-only}
To begin, consider the simplified case in which only excess galaxies contribute to IA. This is given by equation 4.11 of Blazek et al. 2012, adjusted to consider only a single effective lens redshift:
\begin{equation}
\bar{\gamma}_{\rm IA}(r_p) = -\frac{\int dz_s \frac{dN}{dz_s} \xi_{l+}(r_p, \Pi(z_s); z_l)}{\int dz_s \frac{dN}{dz_s} \xi_{ls}(r_p, \Pi(z_s); z_l)} \approx \frac{w_{l+}(r_p)}{w_{ls}(r_p)}.
\label{gamma_blazek}
\end{equation}
This is given in terms of spectroscopic source redshifts only, because $\gamma_{\rm IA}(r_p)$ is already fully corrected for photometric effects by the inclusion of $N_{\rm corr}$. Note that weights, being independent of redshift, have canceled. 

Equation \ref{gamma_blazek} can be heuristically understood as a ratio of the projected IA tangential shear from excess galaxies and the projected number of pairs of excess source and lens galaxies. However, equation \ref{gamma_blazek} does not strictly follow the typical definition of a projected correlation function, which is given by (see, eg., Singh 2014, equation 7):
\begin{equation}
w_{\rm AB}(r_p) = \int dz W(z) \int d\Pi \xi_{\rm AB}(r_p, \Pi, z)
\label{wABW}
\end{equation}
where 
\begin{equation}
W(z) = \left(\frac{\frac{dN^l}{dz}\frac{dN^s}{dz}}{\chi(z)^2 (d\chi/dz)}\right) \left(\int dz \frac{\frac{dN^l}{dz}\frac{dN^s}{dz}}{\chi(z)^2 (d\chi/dz)}\right)^{-1}.
\label{wz}
\end{equation}
In the limit of an effect lens redshift, equation \ref{wABW} becomes:
\begin{equation}
w_{\rm AB}(r_p) = \int d\Pi \xi_{\rm AB}(r_p, \Pi, z_l).
\label{wAB}
\end{equation}
It can be seen that the numerator and denominator of equation \ref{gamma_blazek} each have an additional factor of $\frac{dN}{dz_s}\frac{dz_s}{d\chi}$ (the second of which results from changing variables from $dz_s$ to $d\Pi$). {\it I assume that because this is present in both numerator and denominator, it approximately cancels, and that this is responsible for the approximation sign in equation \ref{gamma_blazek}.} 

\subsubsection*{2-halo terms}
We proceed with computing $\gamma_{\rm IA}(r_p) \approx w_{l+} / w_{ls}$ using the definition of the projected correlation function in the form of equation \ref{wAB}. From equations 8 and 9 of Singh et al. 2015, adjusted to neglect RSD, we get:
\begin{align}
w_{ls}^{2h}(r_p) &= \frac{b_S b_l}{\pi^2} \int dk_z  \int dk_\perp  \frac{k_\perp}{k_z} P^{HF}(\sqrt{k_\perp^2 + k_z^2}, z_l) \sin(k_z \Pi_{\rm max}) J_0(r_p k_\perp) \nonumber \\
w_{l+}^{2h}(r_p) &= \frac{A_I b_l C_1 \rho_c \Omega_M}{\pi^2 D(z_l)} \int dk_z  \int dk_\perp  \frac{k_\perp^3 }{(k_z^2 + k_\perp^2)k_z} P^{HF}(\sqrt{k_\perp^2 + k_z^2}, z_l) \sin(k_z \Pi_{\rm max}) J_2(r_p k_\perp)
\label{ws}
\end{align}
where $b_S$ is the bias of the sources and $b_l$ is the bias of the lenses. $P^{HF}(k)$ is the nonlinear matter power spectrum from halofit. {\it Note that this neglects any dependence on weights or photometric redshift - is this okay?} It is important to well-sample $k_\perp$ and $k_z$ for the $w_{ls}$ integral. (Using 10,000 points logarithmically spaced on the range of k$\approx$($10^{-4}$, 2000) gives convergence out to $30$ Mpc/h.)

$\gamma_{\rm IA}(r_p)$ can be obtained in the linear alignment model by invoking these equations with $P(k,z)$ taken as the linear power spectrum. This is not a very good fit on small scales. The slightly better nonlinear alignment model uses the same equations, but allows $P(k,z)$ to be the nonlinear matter power spectrum as produced by halofit. This produces a reasonable prediction to smaller scales (depsite being known to neglects some terms which should be important). 

At the smallest scales in the regime we care about, we need to introduce a 1-halo term, both the standard one (for $w_{ls}(r_p)$) and from the halo model for intrinsic alignments (for $w_{l\gamma}(r_p)$). 

\subsubsection*{1-halo terms}
To get accurate 2-point functions at small scales, we need to include 1-halo terms in $w_{ls}$ and $w_{l+}$. 

$w_{ls}(r_p)$ requires the standard galaxy-galaxy 1-halo term. To compute this, we assume an NFW profile, given by:
\begin{equation}
\rho(r, M, z) = \frac{\rho_s(M,z)}{\frac{c_{\rm vir}(M,z)r}{R_{\rm vir}(M,z)}\left(1+\frac{c_{\rm vir}(M,z)r}{R_{\rm vir}(M,z)}\right)^2}.
\label{nfw}
\end{equation}
There are several subcomponents here which require a definiton. $\rho_s$ is given by
\begin{equation}
\rho_s(M,z) = \frac{Mc_{\rm vir}(M,z)}{4 \pi R_{\rm vir}(M,z)}\left[{\rm ln} (1+ c_{\rm vir}(M,z)) - \frac{c_{\rm vir}(M,z)}{1+c_{\rm vir}(M,z)}\right]^{-1}.
\label{rhos}
\end{equation}
$R_{\rm vir}(M,z)$ is the virial radius, which can be defined in a number of different ways. At the moment we have followed, e.g., Mandelbaum 2006:
\begin{equation}
R_{\rm vir}(M) = \left(\frac{3 M_{\rm vir}}{4 \pi 180 \rho_c \Omega_{\rm M}} \right)^{\frac{1}{3}}.
\label{Rvir}
\end{equation}
$c_{\rm vir}(M,z)$ is a concentration parameter. There are several empirical fit to its scaling with mass; at the moment we are using:
\begin{equation}
c_{\rm vir}(M) = 5\left(\frac{M}{10^{14}h^{-1}M_{\odot}}\right)^{-0.1}.
\label{cvir}
\end{equation}

Given then the NFW mass profile, we can compute its Fourier transform, $u(k, M)$:
\begin{equation}
u(k, M) = \int_0^{R_{\rm vir}} dr \frac{4 \pi r^2}{M}\frac{\sin(kr)}{kr}\rho(r, M).
\label{u}
\end{equation}

We must account for both the `central-satellite' term and the `satellite-satellite' term of the 2-point spectrum. The 1-halo power spectrum which accounts for both of these terms is (taken from D. Grin, Thesis) is:
\begin{equation}
P_{gg}^{1h}(k) = \frac{n_H}{n_g^2}\left[\langle N_c N_s \rangle u + \langle N_s^2 \rangle u^2\right]
\label{1hpgg}
\end{equation}
where $n_H$ is the volume density of halos at mass $M$, and $n_g$ is the volume density of galaxies in the survey. $\langle N_c N_s \rangle$ is the variance in the number of central-satellite pairs in a halo of mass $M$, while $\langle N_s^2 \rangle$ is the variance in the number of satellite-satellite pairs. (I have specified to a single average mass $M$ here.) 

For Poisson statistics, we have that $\langle N^2 \rangle = \langle N \rangle^2$. However, there may be some deviation from Poisson statistics for small mass halos, which may be encapsulated by introducing a parameter $\alpha$ such that $\langle N^2 \rangle = \alpha^2\langle N \rangle^2$. $\alpha$ is given by:
\begin{equation}
\alpha = \log\left(\frac{M}{10^{11}M_\odot h^{-1}}\right)^{0.5}
\label{alpha}
\end{equation}
if $M$ is less than $10^{11}$ in units of $M_\odot / h$, and $\alpha=1$ otherwise. 

We then use this property of Poisson (or quasi-Poisson) statistics to get $\langle N_c N_s \rangle$ and $\langle N_s^2 \rangle$ in terms of $\langle N_c \rangle$ and $\langle N_s \rangle$. We obtain these latter quantities via the observationally-determined satellite galaxy fraction, which we will call $f_s$. From Reid \& Spergel 2008, we take $f_s = 0.0636$ for SDSS LRGs. We also define the central galaxy fraction, $f_c$, where $f_c + f_s=1$. 

We will make the assumption that there is always one central galaxy per halo. The mean number of central galaxies per halo is hence $\langle N_c \rangle=1$. We can work out the expression for $\langle N_s \rangle$:
\begin{equation}
\langle N_s \rangle = \frac{f_s N_g}{N_H} = \frac{f_s N_g}{f_c N_g} = \frac{f_s}{f_c} = \frac{f_s}{1-f_s}
\label{Nsmean}
\end{equation}
where $N_H$ is the number of halos, $N_g$ is the total number of galaxies, and the third equality follows from the above assumption that the number of halos is equal to the number of central galaxies.

Having now defined $f_c$ and $f_s$, we can also simplify the prefactor of $\frac{n_H}{n_g^2}$ on equation \ref{1hpgg} by taking advantage of our assumption of a single central galaxy per halo. We have:
\begin{equation}
\frac{n_H}{n_g^2} = \frac{f_c n_g}{n_g^2} = \frac{f_c}{n_g} = \frac{1- f_s}{n_g}
\label{prefac1hpgg}
\end{equation}
such that the fully simplified expression for the 1-halo galaxy-galayx power spectrum becomes:
\begin{equation}
P_{gg}^{1h}(k) = \frac{1-f_s}{n_g}\left[\alpha^2 \frac{f_s}{1-f_s} u + \alpha^2 \left(\frac{f_s}{1-f_s}\right)^2 u^2\right].
\label{1hpgg_fs}
\end{equation}
We can then Fourier-transform to get $\xi^{1h}_{ls}(r)$:
\begin{equation}
\xi_{ls}^{1h}(r) = \frac{1}{2\pi^2} \int dk k^2 \frac{\sin(kr)}{kr} P_{gg}^{1h}(k).
\label{xils1h}
\end{equation}
Because in equation \ref{u} we have integrated up to only $R_{\rm vir}$, we have already assumed a lack of inter-halo correlation above the virial radius. Any non-zero values at $r$ greater than this will simply be a result of noise in the Fourier transform, so we set $\xi_{ls}^{1h}(r)$ to zero above $r= R_{\rm vir}$. We then project to get $w_{ls}^{1h}(r_p)$:
\begin{equation}
w_{ls}^{1h}(r_p) = \int_{-\Pi_{\rm max}}^{\Pi_{\rm max}} \xi_{ls}^{1h}(\sqrt{r_p^2 + \Pi^2}).
\label{wl1h}
\end{equation}

$w_{g+}(r_p)$ also requires a 1-halo term, this time given by the halo model for intrinsic alignments. We will calculate $w_{g+}^{1h}(r_p)$ separately, and do $w_{g+}^{\rm tot}(r_p) = w_{g+}(r_p) + w_{g+}^{1h}(r_p)$. We follow Singh et al. 2014 and Schneider \& Bridle 2010 in computing the following.

Schneider and Bridle 2010 give a fitting formula for the 1-halo IA term as:
\begin{equation}
P^{1h}_{g\gamma}(k,z) = a_h \frac{(k/p_1)^2}{1+ (k/p_2)^{p_3}}
\label{P1hIA}
\end{equation}
where
\begin{equation}
p_i = q_{i1} {\rm exp}(q_{i2} z^{q_{i3}}).
\label{pi}
\end{equation}
The parametes $q_{ij}$ are provided in both Schneider \& Bridle 2010 and Singh et al. 2014 for their respective galaxy samples. {\it At the moment we are finding that we have to set $a_h=1$ in the Singh et al. 2014 case although this is not the value provided in the paper in order to obtain results matching what is plotted in that work.}

Equation 15 of Singh et al. 2014 can be expanded, as well as restricted to a single lens redshift, to obtain:
\begin{equation}
w_{g+}^{1h}(r_p) = \int \frac{dk_\perp}{k_\perp}{2\pi} P_{1h}(k_\perp,z) J_0(r_p k_\perp).
\label{wg1h}
\end{equation}

Once again, we do not expect the inter-halo correlations to be meaningful physical values at much larger separations than the virial radius. In this case, we set $w_{g+}^{1h}(r_p)$ to zero above $r_p = 2 R_{\rm vir}$. We allow for correlations at slightly higher than the virial radius because in this case we have made no inherent assumption that the halo ends sharply at $R_{\rm vir}$, and we physically expect some correlation above this value. The choice of $r_p = 2 R_{\rm vir}$ also allows for non-zero correlation out to $1$ Mpc/h, the point to which we expect significant contributions from the 1-halo term. Finally, $r_p$ is a projected radius while the virial radius is a 3D separation, so allowing this flexibility avoids inadvertently cutting inter-halo structure by setting $w_{g+}^{1h}(r_p)$ to zero strictly above $r_p = R_{\rm vir}$. {\it This choice of cutoff is a bit ad-hoc.}

\subsubsection*{Random + excess}
The random and excess galaxies are affected in the same way by intrinsic alignments, so there is no need to alter the numerator of equation \ref{gamma_blazek}.
 We do, however, need to add the random contribution to the number of lens-source pairs in the denominator. From equations 4.8 and 4.10 of Blazek et al. 2012, it can be deduced that the sum over random pairs, in case of perfect spectroscopic redshifts and an effective lens redshift, is given by:
\begin{equation}
\int dz_s \frac{dN}{dz_s} 
\end{equation}
where I have dropped the weights as they will cancel. Adding this to the sum over excess, the denominator of equation \ref{gamma_blazek} becomes
\begin{equation}
\int dz_s \frac{dN}{dz_s} \left(\xi_{ls}(r_p, \Pi(z_s), z_l) + 1\right).
\label{denom_rand}
\end{equation}
This is intuitively sensible. We expect the number of lens-source pairs from excess galaxies to vary as a function of $r_p$ due to clustering, but we expect that from random galaxies to be independent of separation, other than a slight dependence on $z_s$ via $\Pi$. Mapping this to the form of equation \ref{wAB}, we find:
\begin{equation}
\int d\Pi (\xi_{ls}(r_p, \Pi, z_l) + 1) = w_{ls}(r_p) + 2\Pi_{\rm max}.
\label{wl}
\end{equation}
The result is that equation \ref{gamma_blazek} is modified to be:
\begin{equation}
\gamma_{\rm IA}(r_p) \approx \frac{w_{l+}(r_p)}{w_{ls}(r_p) + 2 \Pi_{\rm max}}.
\label{gamma_blazek_randoms}
\end{equation} 


\section*{Total covariance matrix}

The diagonal elements of the total covariance matrix are:
\begin{equation}
C(r_p,r_p) = \sigma^2(r_p) + S_{dN}^2(r_p, r_p) + S_{p}^2(r_p, r_p)
\label{diag}
\end{equation}
and the off-diagonal elements are given by
\begin{equation}
C(r_p, r_p') = S_{dN}^2(r_p,r_p')+S_{p}^2(r_p,r_p')
\label{off_diag}
\end{equation}
{\it under the likely-incorrect assumption that the statistical error from the boost has no off-diagonal elements}.

\section*{Factors which affect errors}
There are numerous degrees of freedom in the calculation which can affect the calculated errors on $(1-a)\gamma_{\rm IA}$. We break them up into several categories.

Directly related to the covariance matrix:
\begin{itemize}
\item{$e_{\rm rms}$ for each shape measurement method}
\item{the density of lens galaxies}
\item{the density of source galaxies}
\item{covariance of $\gamma_t$ between the two shape-measurement methods}
\item{the level of systematic error on $\frac{1}{N_{\rm corr}}$}
\item{fiducial model / values for $\gamma_{\rm IA}$, $a$, and $\frac{dN}{dz_s}$}
\item{the photo-z model}
\end{itemize}

Related to redshift setup (and so indirectly related to the covariance matrix):
\begin{itemize}
\item{the redshift of the lenses}
\item{the line-of-sight separation from the lens within which we assume galaxies are subject to IA}
\item{Cosmological parameters (affect $z$ to $\chi$ conversion)}
\end{itemize}

Currently no effect but could have an effect if weights were redshift-dependent:
\begin{itemize}
\item{the signal to noise (affects $\sigma_e$)}
\item{the intermediate $e_{\rm rms}$ which goes into the weights}
\end{itemize}

\section*{Questions}
\begin{enumerate}
\item{Is using the weighted-fraction for getting $N_{ls}$ generally okay or just in this case where the weights are z-independent? }
\item{The definition of fiducial $\gamma_{\rm IA}$ from Blazek 2012 says it is approximately equal to a ratio of projected correlation functions, but this is using a different definition than the standard definition for correlation functions. Is this where the `approximately' comes in?}
\item{How to deal with off-diagonal elements of the contribution to the statistical error from the boost?}
\end{enumerate}
%-------------------------------------------------------------------------------


%-------------------------------------------------------------------------------

\end{document}
