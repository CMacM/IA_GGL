\documentclass[onecolumn,amsmath,aps,fleqn, superscriptaddress]{revtex4}
 
\usepackage{amssymb}
\usepackage{amsmath}
\usepackage{epsfig}
\usepackage{subfigure}
\usepackage{mathrsfs}
\usepackage{longtable}
\usepackage{enumerate} 
\usepackage{multirow}
\usepackage{color}

\usepackage[usenames,dvipsnames]{xcolor}
\usepackage{hyperref}
\hypersetup{
    colorlinks = true,
    citecolor = {MidnightBlue},
    linkcolor = {BrickRed},
    urlcolor = {BrickRed}
}

\newcommand{\be}{\begin{eqnarray}}
\newcommand{\ee}{\end{eqnarray}}

\allowdisplaybreaks[1]
 
\begin{document}

\title{Intrinsic alignments from different shape measurement techniques}

\author{Danielle Leonard}

\maketitle

\paragraph*{Background} Different shape measurement methods are sensitive to different parts of the radial profile of a galaxy. In Singh et al. 2016, it appears as though the intrinsic alignment signals measured by these different methods are offset from one another by a constant. By measuring a correlation between galaxy position and shape with two different shape measurements, and subtracting these, you could get rid of the lensing signal and obtain a constraint on the scale-dependence of intrinsic alignments. This is useful for building intrinsic alignment models.

\paragraph*{Source sample} This method naturally allows us to loosen the standard assumption that only excess galaxies are subject to IA. We consider galaxies which form part of the smooth source distribution, but which are still physically close to a lens galaxy, to be impacted by intrinsic alignments as well. We assume a source sample which is defined by the requirement that the photo-z of a source galaxy is always within a certain cut-off separation of the lens galaxy in question. In the case of an effective redshift for the lenses, call this cut $z_{\rm cut}$.

We write:
\begin{align}
\tilde{\gamma}(r_p) &= \frac{\sum\limits_{k}^{\rm lens} \tilde{w}_k \tilde{\gamma}_k}{\sum\limits_{k}^{\rm lens} \tilde{w}_k} = \frac{\sum\limits_{k}^{\rm lens} \tilde{w}_k \gamma_{\rm L}^k}{\sum\limits_{k}^{\rm lens} \tilde{w}_k}+\frac{\sum\limits_{k}^{\rm lens} \tilde{w}_k \gamma_{\rm IA}^k}{\sum\limits_{k}^{\rm lens} \tilde{w}_k} \nonumber \\ 
\tilde{\gamma}'(r_p) &= \frac{\sum\limits_{k}^{\rm lens} \tilde{w}_k \tilde{\gamma}_k}{\sum\limits_{k}^{\rm lens} \tilde{w}_k} = \frac{\sum\limits_{k}^{\rm lens} \tilde{w}_k \gamma_{\rm L}^k}{\sum\limits_{k}^{\rm lens} \tilde{w}_k}+\frac{\sum\limits_{k}^{\rm lens} \tilde{w}_k a \gamma_{\rm IA}^k}{\sum\limits_{k}^{\rm lens} \tilde{w}_k} 
\label{full_both}
\end{align}
where all shear should be assumed to be tangential about the lens galaxy. When computing averaged tangential shear about a lens galaxy, the inverse-variance weights are:
\begin{equation}
\tilde{w}_k = \frac{1}{e_{\rm rms}^2 + (\sigma_e^k)^2}
\label{weights}
\end{equation}

A couple of notes on equation \ref{full_both}:
\begin{enumerate}
\item{The optimal choice for the two-point function here is the averaged tangential shear rather than $\Delta \Sigma$. $\Delta \Sigma$ is ill-defined for source galaxies which are very near in redshift to the lens galaxy or in front of the lens galaxy. By definition $\Delta \Sigma$ will be ill-defined for much of our sample.}
\item{The weights in equation \ref{full_both} are the same for both shape measurement methods. This is not an obvious thing to choose, because $e_{\rm rms}$ would be expected to vary between the methods. But, it's important to choose the weights the same so that lensing will cancel when we do the subtraction. Also, it turns out that because $e_{\rm rms}$ dominates over measurement error, choosing a non-optimal weight by changing $e_{\rm rms}$ has little effect on how optimal the weights are, because we are basically weighting every pair the same regardless. So, for the weights, we pick some $e_{\rm rms}$ which is between the values for the two shape measurement methods.}
\item{The boost factor is not included in equation \ref{full_both}. This is because the point of the boost factor is that if you are trying to get a measurement of lensing by summing tangential shears, then you need the boost factor or else your measured signal will be low. But here, we are not trying to get the lensing signal. We are interested in the intrinsic alignment signal, and the lensing signal will cancel with or without the boost, so it is not necessary.}
\end{enumerate}

We subtract the above relationships for the two shape measurement methods to get:
\begin{align}
\tilde{\gamma}(r_p) - \tilde{\gamma}^\prime(r_p) &= \frac{\sum\limits_{k}^{\rm lens} \tilde{w}_k \gamma_{\rm IA}^k}{\sum\limits_{k}^{\rm lens} \tilde{w}_k} - \frac{\sum\limits_{k}^{\rm lens} \tilde{w}_k a \gamma_{\rm IA}^k}{\sum\limits_{k}^{\rm lens} \tilde{w}_k} 
\label{subtract_gammat}
\end{align}
In principle, all the galaxies in our source sample contribute to intrinsic alignments, so we can consider an average per-galaxy contribution of $\gamma_{\rm IA}(r_p)$.
\begin{equation}
\tilde{\gamma}(r_p) - \tilde{\gamma}^\prime(r_p) = \gamma_{\rm IA}(r_p)(1-a).
\label{subtract_gammat_simp}
\end{equation}
However, there is another factor here that we have not yet accounted for. Because we are assuming photometric redshifts for our sources, we will have source galaxies in our sample which are actually higher redshift than $z_{\rm cut}$. These galaxies will not be affected by intrinsic alignments. Therefore, if we apply equation \ref{subtract_gammat_simp} directly, we will estimate the intrinsic alignment signal to be too low. We need to apply a correction factor which accounts for this. 

\subsection*{Correction factor: inclusion of higher-z galaxies}
We assume that we know the smooth distribution of source galaxies, $dN / dz$, to some reasonably high redshift, from a spectroscopic subsample. This may be difficult for future surveys, in which an inadequate spectroscopic subsample, especially at high redshifts, could result in systematic error. {\it We will need to account for this possible systematic error at some point.}

Let the probability of labelling a galaxy with spectroscopic redshift $z_s$ as having photometric redshift $z_p$ be $p(z_s, z_p)$. {\it I think} the total weighted number of galaxies in the photo-z range between $z_l$ and $z_{\rm cut}$ is:
\begin{equation}
N_{\rm tot} = \int_{z_l}^{z_{\rm cut}} dz_p \int_{z_l}^{z_{\rm max}} \frac{dN}{dz_s} p(z_s, z_p) \tilde{w}(z_p)
\label{tot}
\end{equation}
and the number of galaxies in the photo-z range between $z_l$ and $z_{\rm cut}$ which have $z_s$ above $z_{\rm cut}$ is:
\begin{equation}
N_{\rm high} = \int_{z_l}^{z_{\rm cut}} dz_p \int_{z_{\rm cut}}^{z_{\rm max}} \frac{dN}{dz_s} p(z_s, z_p) \tilde{w}(z_p).
\label{high}
\end{equation}
In this case the weights are independent of redshift but they could in principle depend on redshift.

To correct for higher-z galaxies being included, we alter equation \ref{subtract_gammat_simp} as follows:
\begin{equation}
\tilde{\gamma}(r_p) - \tilde{\gamma}^\prime(r_p) = \gamma_{\rm IA}(r_p)(1-a) \left(\frac{N_{\rm tot} - N_{\rm high}}{N_{\rm tot}}\right)
\label{subtract_gammat_fac}
\end{equation}
which is trivially rearranged to solve for the IA term:
\begin{equation}
\gamma_{\rm IA}(r_p)(1-a) = \left(\tilde{\gamma}(r_p) - \tilde{\gamma}^\prime(r_p)\right) \left(\frac{N_{\rm tot}}{N_{\rm tot} - N_{\rm high}}\right)
\label{IAsol}
\end{equation}

To estimate the statistical error on the left hand side of equation \ref{IAsol}, we use the expression:
\begin{equation}
\sigma^2(\gamma_{\rm IA}(r_p)(1-a)) = \left(\frac{N_{\rm tot}}{N_{\rm tot} - N_{\rm high}}\right) \sigma^2\left(\tilde{\gamma}(r_p) - \tilde{\gamma}^\prime(r_p)\right).
\label{error}
\end{equation}
The error on the averaged tangential shear obtained using each shape-measurement method is dominated by shape noise, so the covariance can be represented by a diagonal matrix with $e_{\rm rms}^2 / N_{ls}$ on the diagonal for each bin and the correct choice of $e_{\rm rms}$ for each shape measurement. The errors for the two shape measurements are correlated so this must be taken into account when combining them. The correction factor contributes negligible statistical error so is treated as a constant.

\section*{Questions}
\begin{enumerate}
\item{Does the correction factor seem sensible when calculated?}
\item{What are the properties of the systematic error from not knowing $dN/dz$ perfectly?}
\end{enumerate}
%-------------------------------------------------------------------------------


%-------------------------------------------------------------------------------

\end{document}
