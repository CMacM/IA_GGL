\documentclass[onecolumn,amsmath,aps,fleqn, superscriptaddress]{revtex4}
 
\usepackage{amssymb}
\usepackage{amsmath}
\usepackage{epsfig}
\usepackage{subfigure}
\usepackage{mathrsfs}
\usepackage{longtable}
\usepackage{enumerate} 
\usepackage{multirow}
\usepackage{color}

\usepackage[usenames,dvipsnames]{xcolor}
\usepackage{hyperref}
\hypersetup{
    colorlinks = true,
    citecolor = {MidnightBlue},
    linkcolor = {BrickRed},
    urlcolor = {BrickRed}
}

\newcommand{\be}{\begin{eqnarray}}
\newcommand{\ee}{\end{eqnarray}}

\allowdisplaybreaks[1]
 
\begin{document}

\title{Forecasting errors on $\bar{\gamma}^{\rm IA}(r_p)$, dropping assumption that only excess galaxies contribute}

\author{Danielle Leonard}

\maketitle

\section{Statistical errors}

The expression for $\bar{\gamma}^{\rm IA}(r_p)$ is:
\begin{equation}
\bar{\gamma}^{\rm IA}(r_p) = \frac{c_z^a \widetilde{\Delta\Sigma}^a(r_p) - c_z^b \widetilde{\Delta\Sigma}^b(r_p)}{(B^a(r_p)-1+F^a)c_z^a \langle \tilde{\Sigma}_c\rangle_{\rm IA}^a(r_p) -(B^b(r_p)-1+F^b)c_z^b \langle \tilde{\Sigma}_c\rangle_{\rm IA}^b(r_p)}
\label{gamma_sol_oct26}
\end{equation}

The statistical errors which affect $\bar{\gamma}^{\rm IA}(r_p)$ are:
\begin{itemize}
\item{error on $\widetilde{\Delta\Sigma}(r_p)$, which is dominated by shape noise (uncorrelated between sample $a$ and $b$).}
\item{error on $(B(r_p)-1+F)$. This is essentially an error on $B$ and is sourced at small scales by shot noise and at large scales by cosmic variance.}
\end{itemize}
These are independent: the $\widetilde{\Delta\Sigma}(r_p)$ errors are due to shape noise, which doesn't affect $(B(r_p)-1+F)$.

{\it N.B.:} There is no reason to believe that the covariance matix of $(B(r_p)-1+F)$ in projected radial bins should be zero on off-diagonal elements (which is the case for the covariance matrix of $\widetilde{\Delta\Sigma}(r_p)$). Combining the off-diagonal elements of these contributions is more difficult than combining diagonal elements, because they are covariances rather than variances. For now, we will examine only the diagonal elements of the statistical error. 

We can write the statistical error on $\bar{\gamma}^{\rm IA}$ in a given $r_p$ bin as:
\begin{align}
\sigma^2(\bar{\gamma}^{\rm IA}) &= (\bar{\gamma}^{\rm IA})^2 \left[\frac{\sigma^2\left(c_z^a \widetilde{\Delta\Sigma}^a - c_z^b \widetilde{\Delta\Sigma}^b\right)}{(c_z^a \widetilde{\Delta\Sigma}^a - c_z^b \widetilde{\Delta\Sigma}^b)^2} + \frac{\sigma^2\left((B^a-1+F^a)c_z^a \langle \tilde{\Sigma}_c\rangle_{\rm IA}^a - (B^b-1+F^b)c_z^b \langle \tilde{\Sigma}_c\rangle_{\rm IA}^b\right)}{((B^a-1+F^a)c_z^a \langle \tilde{\Sigma}_c\rangle_{\rm IA}^a - (B^b-1+F^b)c_z^b \langle \tilde{\Sigma}_c\rangle_{\rm IA}^b)^2} \right] 
\label{siggam}
\end{align}
where
\begin{equation}
\sigma^2(c_z^a \widetilde{\Delta\Sigma}^a  - c_z^b \widetilde{\Delta\Sigma}^b) = (c_z^a)^2  \sigma^2 (\widetilde{\Delta\Sigma}^a) + (c_z^b)^2\sigma^2(\widetilde{\Delta\Sigma}^b)
\label{sigtop}
\end{equation}
and 
\begin{equation}
\sigma^2((B^a-1+F^a)c_z^a \langle \tilde{\Sigma}_c\rangle_{\rm IA}^a - (B^b-1+F^b)c_z^b \langle \tilde{\Sigma}_c\rangle_{\rm IA}^b) = (c_z^a \langle \tilde{\Sigma}_c\rangle_{\rm IA}^a)^2 \sigma^2(B^a-1+F^a) + (c_z^b \langle \tilde{\Sigma}_c\rangle_{\rm IA}^b)^2 \sigma^2(B^b-1+F^b).
\label{sigbottom}
\end{equation}

There are several quantities for which we must make some reasonable calculation in a fiducial case: $c_z$, ($B(r_p)-1+F$), $\langle \tilde{\Sigma}_c\rangle_{\rm IA}(r_p)$, $\widetilde{\Delta\Sigma}(r_p)$, and $\bar{\gamma}^{\rm IA}(r_p)$, as well as $\sigma(\widetilde{\Delta \Sigma})$ and $\sigma(B-1+F)$.

\subsection{$\sigma(B(r_p)-1+F)$}
We will assume that the error from factors of $(B-1+F)$ is dominated by error on $(B-1)$, because the statistical error on $F$ can be made arbitrarily small (due to the fact that it depends on `random' pairs only). Therefore we in fact use:
\begin{equation}
\sigma^2((B^a-1+F^a)c_z^a \langle \tilde{\Sigma}_c\rangle_{\rm IA}^a - (B^b-1+F^b)c_z^b \langle \tilde{\Sigma}_c\rangle_{\rm IA}^b) = (c_z^a \langle \tilde{\Sigma}_c\rangle_{\rm IA}^a)^2 \sigma^2(B^a-1) + (c_z^b \langle \tilde{\Sigma}_c\rangle_{\rm IA}^b)^2 \sigma^2(B^b-1).
\label{sigbottom}
\end{equation}
The variance of the quantity $(B^i-1)$ is at the moment input directly from file as computed by Rachel for each sample.

\subsection{$\sigma(\widetilde{\Delta \Sigma}(r_p))$}
Following equation 12 of Mandelbaum et al. 2008 (0709.1692), the variance of $\widetilde{\Delta\Sigma}$ is given by:
\begin{equation}
\sigma^2(\widetilde{\Delta\Sigma}) = \frac{\sum\limits_i^{ls} \tilde{w}_i^2 \tilde{\Sigma}_{c,i}^2 (\epsilon_{\rm rms}^2 + \sigma_i^2)}{\left(\sum\limits_i^{ls} \tilde{w}_i\right)^2}
\label{varDS}
\end{equation}
where $\epsilon_{\rm rms}$ is the root-mean-squared ellipticity of the source sample, $\sigma_i$ is the measurement error for each source, and:
\begin{equation}
\tilde{w}_i = \frac{1}{\Sigma_{c,i}^2(\epsilon_{\rm rms}^2 + \sigma_i^2)}
\label{weights}
\end{equation}

Given this expression for the weights, equation \ref{varDS} simplifies to:
\begin{equation}
\sigma^2(\widetilde{\Delta\Sigma}) = \frac{\sum\limits_i^{ls} \tilde{w_i}}{\left(\sum\limits_i^{ls} \tilde{w_i}\right)^2} = \frac{1}{\sum\limits_i^{ls} \tilde{w_i}}.
\label{varDS_2}
\end{equation}
The above sum over weights is over all lens-source pairs (random + excess). We can re-write into a sum over random using the boost factor:
\begin{equation}
\sigma^2(\widetilde{\Delta\Sigma}(r_p)) = \frac{1}{B(r_p)\sum\limits_i^{{\rm rand} \, ls} \tilde{w_i}}.
\label{varDS_2}
\end{equation}
There are other instances where we compute this sum over weights, below. However, in this case, we need to be particularly careful about the normalization of this sum. All over sums over weights are taken as a ratio with a sum of something involving the weights and normalized in the same way, so the normalization is somewhat arbitrary. Here, it must be correct.

Converting sums to integrals and incorporating this normalization, we find:
\begin{equation}
\sigma^2(\widetilde{\Delta\Sigma}(r_p^i)) = \frac{1}{B(r_p^i)} \left(\frac{N_l N_s^i\int_{z_{\rm cut_l}}^{z_{\rm cut_h}} dz_p \tilde{w}(z_p, z_l) \int_{z_{\rm min}}^{z_{\rm max}} dz_s \frac{dN}{dz_s}p(z_s, z_p)}{ \int_{z_{\rm min}}^{z_{\rm max}} dz_p \int_{z_{\rm min}}^{z_{\rm max}} dz_s \frac{dN}{dz_s}p(z_s, z_p)}\right)^{-1}.
\label{sum_varDS}
\end{equation}
where $N_l$ is number of lenses and $N_s^i$ is the number of sources in projected radial separation bin $i$, obtained via the surface density of sources. 

Note that it is in fact consistent to use the number of density of sources for the full sample to compute $N_s^i$ even though this is used in normalizing a sum over randoms (rather than excess + randoms). This is because when the surface density is computed, it is obtained for the full redshift distribution of all the sources, whereas what consitutes an `excess' source galaxy is specific to each lens. All galaxies, excess + random, are included in this surface density value. (If this feels like cheating, it may be helpful to remember that because $B(r_p)$ accounts for correlations due to galaxy clustering, it is less than $1$ on large scales where the correlation is negative.)

\subsection{$B(r_p)-1+F$}

The boost is known to scale like the projected correlation function, which can be modeled as a power law in $r_p$: 
\begin{equation}
(B(r_p)-1) \propto (r_p)^{-0.8}.
\label{boost}
\end{equation}
The amplitude depends on factors including the photometric redshift properties. We include a tunable parameter with value $(B(1 {\rm Mpc/h})-1)$. It is important to note that the boost depends on the accuracy of photometric redshifts, especially for the source sample defined by $z_p>z_l+\Delta z$ (sample `b'). For this sample, if photometric redshifts are very good, the boost may be very near unity. {\it I need to incorporate a method to relate the boost proportionality factor to $\sigma_z$.}

$F$ is given by:
\begin{equation}
F= \frac{\sum\limits_{j}^{\rm rand-close}\tilde{w}_j}{\sum\limits_{j}^{\rm rand} \tilde{w}_j} 
\label{F}
\end{equation}
The denominator of equation \ref{F} can be written as
\begin{equation}
\sum\limits_{j}^{\rm rand} \tilde{w}_j = \frac{\int_{z_{\rm p}^{\rm min}}^{z_{\rm p}^{\rm max}} d z_{\rm p} \tilde{w}(z_l, z_{\rm p}) \int_{z_{\rm s}^{\rm min}}^{z_{\rm s}^{\rm max}} dz_{\rm s} \frac{dN}{dz_{\rm s}} p(z_{\rm s}, z_{\rm p})}{\int_{z_{\rm p}^{\rm min}}^{z_{\rm p}^{\rm max}} d z_{\rm p} \int_{z_{\rm s}^{\rm min}}^{z_{\rm s}^{\rm max}} dz_{\rm s} \frac{dN}{dz_{\rm s}} p(z_{\rm s}, z_{\rm p})}
\label{sumrandweights}
\end{equation}
where $p(z_{\rm s}, z_{\rm p})$ is the probability of measuring a photometric redshift $z_p$ given a spectroscopic source redshift $z_{\rm s}$. $\tilde{w}(z_p)$ represents the weights, given by:
\begin{equation}
\tilde{w}(z_l, z_{\rm p}) = \frac{1}{\tilde{\Sigma}_c^2(z_l, z_{\rm p})(e_{\rm rms}^2 + \sigma_e(z_{\rm p})^2)}.
\label{weights}
\end{equation}
In this context $\sigma_e(z_{\rm p})$ can be assumed to be redshift-independent, and is roughly $2 / (S/N)$, where the signal to noise for the SDSS sample of Blazek et al. is roughly 15.

In equation \ref{sumrandweights} for the sum of weights over random lens-source pairs, the denominator represents a factor which normalizes such that the total number is set to 1. Because we are here working with ratios of such sums, it is actually unimportant whether or not we include this normalizing factor. However, since we have chosen to include it here, it must be included consistently.

The numerator of equation \ref{F} can be written:
\begin{equation}
\sum\limits_{j}^{\rm rand-close}\tilde{w}_j = \frac{\int_{z_{\rm p}^{\rm min}}^{z_{\rm p}^{\rm max}} d z_{\rm p} \tilde{w}(z_l, z_{\rm p}) \int_{z_{\rm s}^{\rm close, min}}^{z_{\rm s}^{\rm close, max}} dz_{\rm s} \frac{dN}{dz_{\rm s}} p(z_{\rm s}, z_{\rm p})}{\int_{z_{\rm p}^{\rm min}}^{z_{\rm p}^{\rm max}} d z_{\rm p} \int_{z_{\rm s}^{\rm min}}^{z_{\rm s}^{\rm max}} dz_{\rm s} \frac{dN}{dz_{\rm s}} p(z_{\rm s}, z_{\rm p})}
\label{sumrandcloseweights}
\end{equation}
where $z_{\rm s}^{\rm close,max}$ and $z_{\rm s}^{\rm close,min}$ are respectively the maximum and minimum redshifts at which we consider sources to be close enough to the lens to contribute to IA. To be abundantly clear, this cut is made on spectroscopic redshifts because it is the physical redshift of the object which determines if it contributes to the IA signal (not the photometric redshift). 

Note that the numerator in particular of $F$ is subject to considerable variation due to changes in the quality of photometric redshift estimation. If the variance of the photo-z distribution is small, the true and photometric redshifts are usually close together. Therefore, in this case, most of the galaxies which have true redshift near enough the lens to be subject to intrinsic alignments will have photo-z within this range as well. Thus, their weights (which are determined by photo-z's) will be heavily suppressed. This means that for a sample defined by $z_l<z<z_l+\Delta z$ (sample `a' here), $F$ is reduced as photo-z improve. $F$ is also reduced as photo-z improve for the sample defined by $z>z_l+\Delta z$, for the simpler reason that with low-variance photometric redshifts, very few galaxies which truly have redshift very near the lens are scattered into the higher-redshift sample.

\subsection{$\langle \tilde{\Sigma}_c\rangle_{\rm IA}(r_p)$}

$\langle \tilde{\Sigma}_c\rangle_{\rm IA}(r_p)$ is given as:
\begin{equation}
\langle \tilde{\Sigma}_c\rangle_{\rm IA} (r_p) =  \frac{\sum\limits_{j}^{\rm excess} \tilde{w}_j \tilde{\Sigma}_{c,j}+ \sum\limits_{j}^{\rm rand-close} \tilde{w}_j \tilde{\Sigma}_{c,j}}{\sum\limits_{j}^{\rm excess} \tilde{w}_j + \sum\limits_{j}^{\rm rand-close} \tilde{w}_j}= \frac{\sum\limits_{j}^{\rm excess} \tilde{w}_j \tilde{\Sigma}_{c,j}+ \sum\limits_{j}^{\rm rand-close} \tilde{w}_j \tilde{\Sigma}_{c,j}}{(B(r_p)-1)\sum\limits_{j}^{\rm rand} \tilde{w}_j + \sum\limits_{j}^{\rm rand-close} \tilde{w}_j}
\label{SigmaIA}
\end{equation} 

We have seen both sums in the denominator of the far right-hand-side before. The sum over `rand' can be computed using equation \ref{sumrandweights}. The sum over `rand-close' is given by equation \ref{sumrandcloseweights}.

The sum of weights and $\tilde{\Sigma}_{c}$ over `rand-close' in the numerator is given by:
\begin{equation}
\sum\limits_{j}^{\rm rand-close} \tilde{w}_j \tilde{\Sigma}_{c,j} = \frac{\int_{z_{\rm p}^{\rm min}}^{z_{\rm p}^{\rm max}} d z_{\rm p} \tilde{w}(z_l, z_{\rm p})\tilde{\Sigma}_c(z_l, z_{\rm p}) \int_{z_{\rm s}^{\rm close, min}}^{z_{\rm s}^{\rm close, max}} dz_{\rm s} \frac{dN}{dz_{\rm s}} p(z_{\rm s}, z_{\rm p})}{\int_{z_{\rm p}^{\rm min}}^{z_{\rm p}^{\rm max}} d z_{\rm p} \int_{z_{\rm s}^{\rm min}}^{z_{\rm s}^{\rm max}} dz_{\rm s} \frac{dN}{dz_{\rm s}} p(z_{\rm s}, z_{\rm p})}.
\label{sumrandcloseweightsSigC}
\end{equation}

Computing the other sum in the numerator requires slightly more caution. The difficulty is in ensuring that it is consistently normalized. The normalization convention we use here (and for most of the calculation, with the exception of computing the covariance matrix) is such that the total number of sources in the sample is equal to 1. This is relativley simple for cases in which we sum over all sources in the sample: we simply divide by the appropriate integral over the number distribution. However, we don't have an equivalent and consistent way of directly normalizing over the number of excess galaxies, because we don't know this number apriori. 

To resolve this issue, first consider that when we have refered to $\tilde{w}$ and $\tilde{w}(z_p)$ in these notes, we have really meant arbitrarily normalized weights. Let us now temporarily call these arbitrarily normalized weights $\tilde{w}^a$ or $\tilde{w}^a(z_p)$, and refer to weigths which are corretly noramlized as $\tilde{w}^n$ or $\tilde{w}^n(z_p)$. We can then write:
\begin{equation}
\sum\limits_{j}^{\rm excess} \tilde{w}^n_j \tilde{\Sigma}_{c,j} = \frac{\sum\limits_{j}^{\rm excess} \tilde{w}^a_j \tilde{\Sigma}_{c,j}}{ \sum\limits_{j}^{\rm excess} \tilde{w}^a_j} \sum\limits_{j}^{\rm excess} \tilde{w}^n_j.
\label{normSigIA1}
\end{equation}
This holds as long as both sets of arbitrarily normalized weights are arbitrarily normalized in the same way. 

We can now compute each of the three terms requires for equation \ref{normSigIA1}. To do so, we assume that all excess source galaxies have spec-z at the lens redshift; this is a reasonable assumption because excess galaxies result from clustering, which is most prominent locally. We therefore consider a similar expression to the numerator of equation \ref{sumrandcloseweightsSigC}, but set $\frac{dN}{dz_s} = \delta(z_s-z_l)$. We have:
\begin{align}
\sum\limits_{j}^{\rm excess} \tilde{w}^a_j \tilde{\Sigma}_{c,j} = \int_{z_{\rm p}^{\rm min}}^{z_{\rm p}^{\rm max}} dz_{\rm p}\tilde{w}^a(z_l, z_{\rm p}) \tilde{\Sigma}_c(z_l, z_{\rm p}) \int_{z_s^{\rm min}}^{z_s^{\rm max}} dz_s \delta(z_s - z_l) p(z_s, z_{\rm p}) = \int_{z_{\rm p}^{\rm min}}^{z_{\rm p}^{\rm max}} dz_{\rm p}\tilde{w}(z_l, z_{\rm p}) \tilde{\Sigma}_c(z_l, z_{\rm p}) p(z_l, z_{\rm p}).
\label{wSigc_atzl}
\end{align}
We can similarly compute the sum over arbitrarily normalized weights of excess sources:
\begin{align}
\sum\limits_{j}^{\rm excess} \tilde{w}^a_j= \int_{z_{\rm p}^{\rm min}}^{z_{\rm p}^{\rm max}} dz_{\rm p}\tilde{w}^a(z_l, z_{\rm p}) \int_{z_s^{\rm min}}^{z_s^{\rm max}} dz_s \delta(z_s - z_l) p(z_s, z_{\rm p}) = \int_{z_{\rm p}^{\rm min}}^{z_{\rm p}^{\rm max}} dz_{\rm p}\tilde{w}(z_l, z_{\rm p})  p(z_l, z_{\rm p}).
\label{excess_w_arb}
\end{align}
The sum over {\it appropriately} normalized weights of excess sources is given by using the relationship which defines the boost factor to turn it into a sum over random sources and normalize in the usual way:
\begin{equation}
\sum\limits_{j}^{\rm excess} \tilde{w}^n_j = (B(r_p)-1) \frac{ \int_{z_{\rm p}^{\rm min}}^{z_{\rm p}^{\rm max}} dz_{\rm p}\tilde{w}^a(z_l, z_{\rm p}) \int_{z_s^{\rm min}}^{z_s^{\rm max}} dz_s \frac{dN}{dz_s} p(z_s, z_{\rm p})}{\int_{z_{\rm p}^{\rm min}}^{z_{\rm p}^{\rm max}} dz_{\rm p} \int_{z_s^{\rm min}}^{z_s^{\rm max}} dz_s \frac{dN}{dz_s} p(z_s, z_{\rm p})}.
\label{excess_w_normed}
\end{equation}
Given then the three components of equation \ref{normSigIA1}, we can compute $\sum\limits_{j}^{\rm excess} \tilde{w}^n_j \tilde{\Sigma}_{c,j}$, and hence $\langle \tilde{\Sigma}_c\rangle_{\rm IA}(r_p)$.

\subsection{$c_z$}

The photometric bias variable $c_z$ is equal to $(1+b_z)^{-1}$.
\begin{equation}
b_z+1 = \frac{B(r_p) \sum\limits^{\rm lens}_{j} \tilde{w}_j \tilde{\Sigma}_{c,j} \Sigma_{c,j}^{-1}}{\sum\limits^{\rm lens}_{j} \tilde{w}_j} = \frac{\sum\limits^{\rm rand}_{j} \tilde{w}_j \tilde{\Sigma}_{c,j} \Sigma_{c,j}^{-1}}{\sum\limits^{\rm rand}_{j} \tilde{w}_j}
\label{photozbias}
\end{equation}
We already have the denominator (equation \ref{sumrandweights}). The numerator can be written as:
\begin{equation}
\sum\limits^{\rm rand}_{j} \tilde{w}_j \tilde{\Sigma}_{c,j} \Sigma_{c,j}^{-1}  = \frac{\int_{z_{\rm p}^{\rm min}}^{z_{\rm p}^{\rm max}}dz_{\rm p} b_{sw}(z_{\rm p}, z_l) \int_{z_{\rm s}^{\rm min}}^{z_{\rm s}^{\rm max}}dz_{\rm s} \frac{dN^s}{dz^s} p(z_{\rm p}, z_{\rm s})}{\int_{z_{\rm p}^{\rm min}}^{z_{\rm p}^{\rm max}}dz_{\rm p}\int_{z_{\rm s}^{\rm min}}^{z_{\rm s}^{\rm max}}dz_{\rm s} \frac{dN^s}{dz^s} p(z_{\rm p}, z_{\rm s})}
\label{photoz}
\end{equation}
where we have defined a factor $b_{sw}(z_{\rm p}, z_l)$. This factor is obtained in the following way:
\begin{itemize}
\item{For a given $z_{\rm p}$, draw from $p(z_{\rm s}, z_{\rm p})$ a number of $z_{\rm s}$ values. These are true-redshift values which could correspond to the measured $z_{\rm p}$.}
\item{For each of these $z_{\rm s}$ values, compute the value of $\tilde{w}(z_{\rm p}, z_l) \tilde{\Sigma}_c(z_{\rm p}, z_l) \Sigma_c(z_{\rm s}, z_l)$.}
\item{Find the mean of these values. This is the value of $b_{sw}(z_{\rm p})$.}
\end{itemize}
For a Gaussian $p(z_{\rm p}, z_{\rm s})$, $b_z+1$ typically ranges from $0.9$ to $0.99$ provided $\sigma_z$ is within the range of what is expected for current and future surveys.

\subsection{$\widetilde{\Delta \Sigma}$}
From equation 3.9 of Blazek et al., the estimated $\widetilde{\Delta \Sigma}$ is given by:
\begin{align}
\widetilde{\Delta \Sigma}&= (1+b_z) \Delta \Sigma + \frac{\sum\limits_{j}^{\rm lens} \tilde{w}_j \tilde{\Sigma}_{c,j} \gamma^{\rm IA}_j }{\sum\limits_{j}^{\rm rand} \tilde{w}_j} 
\label{deltasigma}
\end{align}

\subsubsection{$\Delta \Sigma$, from theory}
The first term of the above is computed using equation \ref{photozbias} and the theoretical value of $\Delta \Sigma$(R). The latter is a projected differential surface density, given by
\begin{equation}
\Delta \Sigma(R) = \frac{2}{R^2} \int_0^R dR' R' \Sigma(R') - \Sigma(R)
\label{DelSigDef}
\end{equation}
where $\Sigma(R)$ is the projected surface density of matter about a galaxy. $\Delta \Sigma(R)$ will be computed by splitting it into two terms:
\begin{equation}
\Delta\Sigma(R) = \Delta\Sigma_{{\rm 1h}}(R) + \Delta \Sigma_{{\rm 2h}}(R).
\label{DeltaSigterm}
\end{equation}
where 1h indicates the 1-halo contribution and 2h the 2-halo contribution.

In the case of the 1-halo term, one must assume a halo density profile, for which we choose NFW:
\begin{equation}
\rho(r, M) = \frac{\rho_s(M)}{\frac{c_{\rm vir}(M)r}{R_{\rm vir}(M)}\left(1+\frac{c_{\rm vir}(M)r}{R_{\rm vir}(M)}\right)^2}.
\label{nfw}
\end{equation}
There are several components here which require a definition. $\rho_s(M,z)$ is given by
\begin{equation}
\rho_s(M) = \frac{Mc_{\rm vir}(M)}{4 \pi R_{\rm vir}(M)}\left[{\rm ln} (1+ c_{\rm vir}(M)) - \frac{c_{\rm vir}(M)}{1+c_{\rm vir}(M)}\right]^{-1}
\label{rhos}
\end{equation}
which follows from the fact that the density integrated over volume out to the virial radius must be equal to the  mass.

$R_{\rm vir}(M)$ is the virial radius, which can be defined in a number of different ways. At the moment we have followed, e.g., Mandelbaum et al. 2006, and set
\begin{equation}
R_{\rm vir}(M) = \left(\frac{3 M_{\rm vir}}{4 \pi 180 \rho_c \Omega_{\rm M}} \right)^{\frac{1}{3}}.
\label{Rvir}
\end{equation}
Note that we are using comoving coordinates, such that densities and hence the virial radius are independent of redshift.

$c_{\rm vir}(M)$ is a concentration parameter. There are several empirical fits to its scaling with mass; at the moment we are using:
\begin{equation}
c_{\rm vir}(M) = 5\left(\frac{M}{10^{14}h^{-1}M_{\odot}}\right)^{-0.1}.
\label{cvir}
\end{equation}

Given these definitions, we can compute the density of the halo using equation \ref{nfw}. The projected density, $\Sigma_{\rm 1h}(R)$, is then given simply by:
\begin{equation}
\Sigma_{1h}(R) = \int_{-\Pi_{\rm max}}^{\Pi_{\rm max}} d\Pi \rho\left(r = \sqrt{R^2 +\Pi^2}\right)
\label{sigma1h}
\end{equation}
where we have suppressed dependence on mass and redshift. 

In the case of the 2-halo term, the projected mass density about the center of the galaxy is given by
\begin{equation}
\Sigma_{\rm 2h}(R) = \rho_c(z) \Omega_{\rm M}(z) \int_{-\Pi_{\rm max}}^{\Pi_{\rm max}} d\Pi \left(1 + \xi_{\rm gm}\left(r = \sqrt{R^2 + \Pi^2}\right)\right)
\label{sigma2h}
\end{equation}
where again dependence on redshift is suppressed. $\xi_{\rm gm}$ is obtained by taking the Fourier transform (using FFTlog) of the nonlinear power spectrum computed via CAMB with halofit. {\it Note that taking this as being equivalent to a 2-halo term neglects the fact that at sufficiently small scales the 2-halo term should vanish because two haloes at such separations would physically be a single halo. To account for this a halo-exclusion term is required, which has not yet been incorporated.} Because $\Delta \Sigma (R)$ involves an average over $R$, the `1' in brackets can be dropped from the integral without affecting the final result (although it does affect the value of $\Sigma_{\rm 2h}(R)$). 

To obtain the combined $\Delta \Sigma(R)$, then, equations \ref{sigma1h} and \ref{sigma2h} are inserted in equation \ref{DelSigDef}, and the results are summed as in equation \ref{DeltaSigterm}.


\subsubsection*{The second term of the RHS of equation \ref{deltasigma}}
The denominator of the second term on the RHS of equation \ref{deltasigma} can be calculated using equation \ref{sumrandweights}. 

To calculate the numerator, first split the sum over all lens-source pairs into a sum over excess and one over randoms:
\begin{equation}
\sum\limits_{j}^{\rm lens} \tilde{w}_j \tilde{\Sigma}_{c,j} \gamma^{\rm IA}_j  = \bar{\gamma}_{\rm IA}(r_p)\left[\sum\limits_{j}^{\rm excess} \tilde{w}_j \tilde{\Sigma}_{c,j}  + \sum\limits_{j}^{\rm rand-close} \tilde{w}_j \tilde{\Sigma}_{c,j} \right]
\label{splitIA}
\end{equation}
where $\gamma_{\rm IA}(r_p)$, the IA shear per contributing pair, has been pulled out of the sum using the same logic as usual. The two terms inside brackets on the RHS can then be calculated using equations \ref{sumexcesswSigC} and \ref{sumrandcloseweightsSigC}, respectively. A fiducial value of $\bar{\gamma}_{\rm IA}(r_p)$ must be computed from some model of intrinsic alignments. 

\subsection{$\bar{\gamma}^{\rm IA}(r_p)$}
\subsubsection*{Excess-only}
To begin, consider the simplified case in which only excess galaxies contribute to IA. This is given by equation 4.11 of Blazek et al. 2012, adjusted to consider only a single effective lens redshift:
\begin{equation}
\bar{\gamma}_{\rm IA}(r_p) = -\frac{\int dz_p \tilde{w}(z_p) \int dz_s \frac{dN}{dz_s} \xi_{l+}(r_p, \Pi(z_s); z_l) p(z_s, z_p)}{\int dz_p \tilde{w}(z_p) \int dz_s \frac{dN}{dz_s} \xi_{ls}(r_p, \Pi(z_s); z_l) p(z_s, z_p)} \approx \frac{w_{l+}(r_p)}{w_{ls}(r_p)}.
\label{gamma_blazek}
\end{equation} 
Equation \ref{gamma_blazek} does not strictly follow the typical definition of projected correlation functions, which is given by (see, eg., Singh 2014, equation 7):
\begin{equation}
w_{\rm AB}(r_p) = \int dz W(z) \int d\Pi \xi_{\rm AB}(r_p, \Pi, z)
\label{wABW}
\end{equation}
where 
\begin{equation}
W(z) = \left(\frac{\frac{dN^A}{dz}\frac{dN^B}{dz}}{\chi(z)^2 (d\chi/dz)}\right) \left(\int dz \frac{\frac{dN^A}{dz}\frac{dN^B}{dz}}{\chi(z)^2 (d\chi/dz)}\right)^{-1}.
\label{wz}
\end{equation}
In the limit of an effect lens redshift, equation \ref{wABW} becomes:
\begin{equation}
w_{\rm AB}(r_p) = \int d\Pi \xi_{\rm AB}(r_p, \Pi, z_l).
\label{wAB}
\end{equation}
It can be seen that the numerator and denominator of equation \ref{gamma_blazek} each have an additional factor of $\frac{dN}{dz_s}\frac{dz_s}{d\chi}$ (the second of which results from changing variables from $dz_s$ to $d\Pi$). {\it I assume that because this is present in both numerator and denominator, it approximately cancels, and that this is responsible for the approximation sign in equation \ref{gamma_blazek}.} 

\subsubsection*{2-halo terms}
We proceed with computing $\gamma_{\rm IA}(r_p) \approx w_{l+} / w_{ls}$ using the definition of the projected correlation function in the form of equation \ref{wAB}. From equations 8 and 9 of Singh et al. 2015, adjusted to neglect RSD, we get:
\begin{align}
w_{ls}^{2h}(r_p) &= \frac{b_S b_l}{\pi^2} \int dk_z  \int dk_\perp  \frac{k_\perp}{k_z} P(\sqrt{k_\perp^2 + k_z^2}, z_l) \sin(k_z \Pi_{\rm max}) J_0(r_p k_\perp) \nonumber \\
w_{l+}^{2h}(r_p) &= \frac{A_I b_l C_1 \rho_c \Omega_M}{\pi^2 D(z_l)} \int dk_z  \int dk_\perp  \frac{k_\perp^3 }{(k_z^2 + k_\perp^2)k_z} P(\sqrt{k_\perp^2 + k_z^2}, z_l) \sin(k_z \Pi_{\rm max}) J_2(r_p k_\perp)
\label{ws}
\end{align}
where $b_S$ is the bias of the sources and $b_l$ is the bias of the lenses. It is important to well-sample $k_\perp$ and $k_z$ for the $w_{ls}$ integral. (Using 10,000 points logarithmically spaced on the range of k$\approx$($10^{-4}$, 2000) gives convergence out to $30$ Mpc/h.)

$\gamma_{\rm IA}(r_p)$ can be obtained in the linear alignment model by invoking these equations with $P(k,z)$ taken as the linear power spectrum. This is not a very good fit on small scales. The slightly better nonlinear alignment model uses the same equations, but allows $P(k,z)$ to be the nonlinear matter power spectrum. This produces a reasonable prediction to smaller scales (depsite being known to neglects some terms which should be important). 

At the smallest scales in the regime we care about, we need to introduce 1-halo terms.

\subsubsection*{1-halo terms}
To get accurate 2-point functions at small scales, we need to include 1-halo terms in $w_{ls}$ and $w_{l+}$. 

$w_{ls}(r_p)$ requires a standard galaxy-galaxy 1-halo term. As before, we assume an NFW profile with definitions for $R_{\rm vir}$, $c_{\rm vir}$, and $\rho_s$ as discussed above. We will then use this to compute $w^{1h}_{ls}(r_p)$, and take $w_{ls} = w_{ls}^{1h} + w_{ls}^{2h}$. {\it Again, halo-exclusion terms are required here.} 

Given the NFW mass profile of equation \ref{nfw}, we can compute its Fourier transform, $u(k, M)$:
\begin{equation}
u(k, M) = \int_0^{R_{\rm vir}} dr \frac{4 \pi r^2}{M}\frac{\sin(kr)}{kr}\rho(r, M).
\label{u}
\end{equation}

We must account for both the `central-satellite' term and the `satellite-satellite' term of the 2-point spectrum. The 1-halo power spectrum which accounts for both of these terms is (taken from D. Grin, Thesis) is:
\begin{equation}
P_{ls}^{1h}(k) = \frac{n_H}{n_g^l n_g^{\rm src}}\left[\left(\langle N_c^l N_s^{\rm src} \rangle + \langle N_c^{\rm src} N_s^{\rm l} \rangle \right) u + \langle N_s^l N_s^{\rm src} \rangle u^2\right]
\label{1hpgg}
\end{equation}
where $n_g^l$ and $n_g^{\rm src}$ are the volume densities of lenses and sources respectively. $\langle N_c N_s \rangle$ is the variance in the number of central-satellite pairs in a halo of mass $M$, while $\langle N_s^2 \rangle$ is the variance in the number of satellite-satellite pairs. $n_H$ is the volume density of halos at mass $M$. Note that for the case in which lens galaxies are significantly more lumnious than source galaxies, $n_H$ should be taken as the halo density associated with the lens sample. This is because while it is expected that halos with central galaxies from the more luminous lens sample may contain satellites from the less luminous source sample, the opposite case is not expected. This effect also means that the $\langle N_c^{\rm src} N_s^{\rm l} \rangle$ will be generally negligible.

For Poisson statistics, we have that $\langle N^2 \rangle = \langle N \rangle^2$. There may be some deviation from Poisson statistics for small mass halos, so we introduce a parameter $\alpha$ such that $\langle N^2 \rangle = \alpha^2\langle N \rangle^2$. $\alpha$ is given by:
\begin{equation}
\alpha = \log\left(\frac{M}{10^{11}M_\odot h^{-1}}\right)^{0.5}
\label{alpha}
\end{equation}
if $M$ is less than $10^{11}$ in units of $M_\odot / h$, and $\alpha=1$ otherwise. We then use this property of Poisson (or quasi-Poisson) statistics to write:
\begin{equation}
P_{ll}^{1h}(k) = \frac{n_H}{n_g^l n_g^{\rm src}}\left[\alpha^2 \left(\langle N_c^l \rangle \langle N_s^{\rm src} \rangle + \langle N_c^{\rm src}\rangle \langle N_s^{\rm l} \rangle\right) u + \alpha^2\langle N_s^l \rangle \langle N_s^{\rm src} \rangle u^2\right]
\label{1hpgg_Pois}
\end{equation}

Let's consider the case in which the lens sample is the SDSS LRGs, and the source sample is the SDSS shape sample. In this case, we can set $\langle N_c^{\rm src}\rangle \langle N_s^{\rm l} \rangle=0$ as discussed above, and $n_H$ is the halo density associated with the LRG lens sample. We will make the assumption that there is always one central galaxy per halo. The mean number of central galaxies per halo is hence $\langle N_c \rangle=1$. We then need to work out the expressions for $\langle N_s^{\rm l} \rangle$ and $\langle N_s^{\rm src} \rangle$.

$\langle N_s^l \rangle$ can be obtained n terms of the satellite fraction of the LRG sample:
\begin{equation}
\langle N_s^l \rangle = \frac{f_s^l N_g^l}{N_H} = \frac{f_s^l N_g^l}{f_c^l N_g^l} = \frac{f_s^l}{f_c^l} = \frac{f_s^l}{1-f_s^l}.
\label{Nsmean_lens}
\end{equation}
where $N_g^l$ is the total number of galaxies in the lens sample, $N_H$ is the total number of halos, $f_s^l$ is the satellite fraction of the lenses, and $f_c^l$ is the corresponding central fraction, where $f_c=1-f_s$.

$\langle N_s^{\rm src} \rangle$ is slightly more complicated. It is equal to the number of galaxies in the source sample {\it which are close enough to the lenses to be within the same halo as a lens galaxy}, divided by the number of halos:
\begin{equation}
\langle N_s^{\rm src} \rangle = \frac{f_{\rm near}^{\rm src} N_g^{\rm src}}{N_H}
\label{Nsmean_sources}
\end{equation}
where $f_{\rm near}^{\rm src}$ is the fraction of the total number of source galaxies $N_g^{\rm src}$ within a halo radius of a lens. We compute $f_{\rm near}^{\rm src}$ as:
\begin{equation}
f_{\rm near}^{\rm src} = \int^{z(\xi_l+R_{\rm vir})}_{z(\xi_l-R_{\rm vir})}dz \frac{dN}{dz_s}
\label{frac_src_near}
\end{equation}
where we have assumed an effective redshift for the lenses corresponding to a comoving distance $\xi_l$, and $\frac{dN}{dz_s}$ is normalized. Noting that $n_H = (1-f_s^l) n_g^l$, we can then write:
\begin{equation}
P_{ls}^{1h}(k) = \frac{1-f_s^l}{n_g^{\rm src}}\left[\alpha^2 \frac{f_{\rm near}^{\rm src}N_g^{\rm src}}{(1-f_s^l)N_g^{l}} u + \frac{f_s^l}{1-f_s^l}\frac{f_{\rm near}^{\rm src}N_g^{\rm src}}{(1-f_s^l)N_g^{l}}u^2\right]
\label{1hpgg_fs}
\end{equation}
We can then Fourier-transform to get $\xi^{1h}_{ls}(r)$:
\begin{equation}
\xi_{ls}^{1h}(r) = \frac{1}{2\pi^2} \int dk k^2 \frac{\sin(kr)}{kr} P_{ls}^{1h}(k).
\label{xils1h}
\end{equation}
Because in equation \ref{u} we have integrated up to only $R_{\rm vir}$, we have already assumed a lack of inter-halo correlation above the virial radius. Any non-zero values at $r$ greater than this will simply be a result of noise in the Fourier transform, so we set $\xi_{ls}^{1h}(r)$ to zero above $r= R_{\rm vir}$. We then project to get $w_{ls}^{1h}(r_p)$:
\begin{equation}
w_{ls}^{1h}(r_p) = \int_{-\Pi_{\rm max}}^{\Pi_{\rm max}} \xi_{ls}^{1h}(\sqrt{r_p^2 + \Pi^2}).
\label{wl1h}
\end{equation}

$w_{g+}(r_p)$ also requires a 1-halo term, this time given by the halo model for intrinsic alignments. We will calculate $w_{g+}^{1h}(r_p)$ separately, and do $w_{g+}^{\rm tot}(r_p) = w_{g+}(r_p) + w_{g+}^{1h}(r_p)$. We follow Singh et al. 2014 and Schneider \& Bridle 2010 in computing the following.

Schneider and Bridle 2010 give a fitting formula for the 1-halo IA term as:
\begin{equation}
P^{1h}_{g\gamma}(k,z) = a_h \frac{(k/p_1)^2}{1+ (k/p_2)^{p_3}}
\label{P1hIA}
\end{equation}
where
\begin{equation}
p_i = q_{i1} {\rm exp}(q_{i2} z^{q_{i3}}).
\label{pi}
\end{equation}
The parametes $q_{ij}$ are provided in both Schneider \& Bridle 2010 and Singh et al. 2014 for their respective galaxy samples. 

Equation 15 of Singh et al. 2014 can be expanded, as well as restricted to a single lens redshift, to obtain:
\begin{equation}
w_{g+}^{1h}(r_p) = \int \frac{dk_\perp}{k_\perp}{2\pi} P_{1h}(k_\perp,z) J_0(r_p k_\perp).
\label{wg1h}
\end{equation}

Once again, we do not expect the inter-halo correlations to be meaningful physical values at much larger separations than the virial radius. In this case, we set $w_{g+}^{1h}(r_p)$ to zero above $r_p = 2 R_{\rm vir}$. We allow for correlations at slightly higher than the virial radius because in this case we have made no inherent assumption that the halo ends sharply at $R_{\rm vir}$, and we physically expect some correlation above this value. The choice of $r_p = 2 R_{\rm vir}$ also allows for non-zero correlation out to $1$ Mpc/h, the point to which we expect significant contributions from the 1-halo term. Finally, $r_p$ is a projected radius while the virial radius is a 3D separation, so allowing this flexibility avoids inadvertently cutting inter-halo structure by setting $w_{g+}^{1h}(r_p)$ to zero strictly above $r_p = R_{\rm vir}$. {\it This choice of cutoff is a bit ad-hoc.}

\subsubsection*{Random + excess}
The random and excess galaxies are affected in the same way by intrinsic alignments, so there is no need to alter the numerator of equation \ref{gamma_blazek}.
 We do, however, need to add the random contribution to the number of lens-source pairs in the denominator. From equations 4.8 and 4.10 of Blazek et al. 2012, it can be deduced that the sum over random pairs, in case of an effective lens redshift, is given by:
\begin{equation}
\int dz_p \tilde{w}(z_p) \int dz_s \frac{dN}{dz_s} p(z_s, z_p).
\end{equation}
Adding this to the sum over excess, the denominator of equation \ref{gamma_blazek} becomes
\begin{equation}
\int dz_p \tilde{w}(z_p) \int dz_s \frac{dN}{dz_s} p(z_s, z_p) \left(\xi_{ls}(r_p, \Pi(z_s), z_l) + 1\right).
\label{denom_rand}
\end{equation}
This is intuitively sensible. We expect the number of lens-source pairs from excess galaxies to vary as a function of $r_p$ due to clustering, but we expect that from random galaxies to be independent of separation, other than a slight dependence on $z_s$ via $\Pi$. Mapping this to the form of equation \ref{wAB}, we find:
\begin{equation}
\int d\Pi (\xi_{ls}(r_p, \Pi, z_l) + 1) = w_{ls}(r_p) + 2\Pi_{\rm max}.
\label{wl}
\end{equation}
The result is that equation \ref{gamma_blazek} is modified to be:
\begin{equation}
\bar{\gamma}_{\rm IA}(r_p) \approx \frac{w_{l+}(r_p)}{w_{ls}(r_p) + 2 \Pi_{\rm max}}.
\label{gamma_blazek_randoms}
\end{equation} 

\section{Systematic errors}
Recall that systematic errors in this case are those which introduce a correlated change in the results between projected $r_p$ bins. The systematic errors which may affect $\gamma_{\rm IA}(r_p)$ measured using this method are:
\begin{itemize}
\item{error as a result of not knowing the source galaxy $\frac{dN}{dz_s}$ perfectly (not having a sufficient representative sample of sources). This propagates into $\langle \tilde{\Sigma}_c\rangle_{\rm IA}$, $F$, and $c_z$.}
\item{error as a result of an imperfect knowledge of the photometric redshift probability distribution $p(z_p, z_s)$. This also propagates into $\langle \tilde{\Sigma}_c\rangle_{\rm IA}$, $F$, and $c_z$.}
\item{error on the boost factor as a result of large-scale structure fluctuations and variations in lens density due to observing conditions. This affects only the boost $B(r_p)$.}
\end{itemize}

Recall that the expression for $\gamma_{\rm IA}(r_p)$ under this method is:
\begin{equation}
\bar{\gamma}^{\rm IA} = \frac{c_z^a \widetilde{\Delta\Sigma}^a - c_z^b \widetilde{\Delta\Sigma}^b}{(B^a-1+F^a)c_z^a \langle \tilde{\Sigma_c}\rangle_{\rm IA}^a -(B^b-1+F^b)c_z^b \langle \tilde{\Sigma_c}\rangle_{\rm IA}^b}.
\label{gamma_sol}
\end{equation}

This expression is complex enough that I think the easiest way to compute the systematic error from each of the three above sources is the following:
\begin{enumerate}
\item{Vary the quantity which sources the systematic error (either $\frac{dN}{dz_s}$, $p(z_p, z_s)$, or $B(r_p)$) and propagate this effect to the corresponding changes on other quantities (if any).}
\item{Compute the change to $\gamma_{\rm IA}(r_p)$ from this systematic error by computing the value of $\gamma_{\rm IA}(r_p)$ with these modified quantities and subtracting the fiducial value of $\gamma_{\rm IA}(r_p)$.}
\end{enumerate}

To be more specific, the systematic error due to uncertainty in $\frac{dN}{dz_s}$ would be:
\begin{align}
\sigma_{dN} &= \frac{(c^a_{z} + \delta c_{N}^{a}) \widetilde{\Delta\Sigma}^a - (c^b_{z} + \delta c_{N}^{b}) \widetilde{\Delta\Sigma}^b}{(B^a-1+F^a + \delta F^a_{N})(c^a_{z} + \delta c_{N}^{a}) (\langle \tilde{\Sigma_c}\rangle_{\rm IA}^a + \delta \langle \tilde{\Sigma_c}\rangle_{N}^{a}) -(B^b-1+F^b + \delta F^b_{N})(c^b_{z} + \delta c_{N}^{b}) (\langle \tilde{\Sigma_c}\rangle_{\rm IA}^b + \delta \langle \tilde{\Sigma_c}\rangle_{N}^{b})} - \gamma_{\rm IA}(r_p).
\label{sigmadNsys}
\end{align}
The systematic error due to uncertainty in the photometric redshift distribution would be:
\begin{align}
\sigma_{pz} &= \frac{(c^a_{z} + \delta c_{p}^{a}) \widetilde{\Delta\Sigma}^a - (c^b_{z} + \delta c_{p}^{b}) \widetilde{\Delta\Sigma}^b}{(B^a-1+F^a + \delta F^a_{p})(c^a_{z} + \delta c_{p}^{a}) (\langle \tilde{\Sigma_c}\rangle_{\rm IA}^a + \delta \langle \tilde{\Sigma_c}\rangle_{p}^{a}) -(B^b-1+F^b + \delta F^b_{p})(c^b_{z} + \delta c_{p}^{b}) (\langle \tilde{\Sigma_c}\rangle_{\rm IA}^b + \delta \langle \tilde{\Sigma_c}\rangle_{p}^{b})} - \gamma_{\rm IA}(r_p).
\label{sigmapzsys}
\end{align}
And the systematic error due to systematic uncertainty in the boost is given by: 
\begin{equation}
\bar{\gamma}^{\rm IA} = \frac{c_z^a \widetilde{\Delta\Sigma}^a - c_z^b \widetilde{\Delta\Sigma}^b}{(B^a +\delta B^a-1+F^a)c_z^a \langle \tilde{\Sigma_c}\rangle_{\rm IA}^a -(B^b + \delta B^b-1+F^b)c_z^b \langle \tilde{\Sigma_c}\rangle_{\rm IA}^b}- \gamma_{\rm IA}(r_p) .
\label{sigmaBsys}
\end{equation}
In all three above cases, quantities other than changes due to systematic uncertainty are calcuated at fiducial values.


\section{Open questions / issues}
\begin{enumerate}
\item{How to treat the off-diagonal elements of the covariance matrix?}
\item{Need to include halo-exclusion term in calculation of $\Delta \Sigma$ and $w_{gg}$.}
\item{The definition of $\gamma_{\rm IA}$ from Blazek et al uses something slightly different than the standard projection function definition - is this okay?}
\item{My equations for $w_{gg}$ and $w_{g+}$ neglect photo-z - is that okay? Taken from Singh 2014.}
\end{enumerate}









%-------------------------------------------------------------------------------


%-------------------------------------------------------------------------------

\end{document}
