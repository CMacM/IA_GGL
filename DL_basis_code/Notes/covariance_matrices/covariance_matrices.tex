\documentclass[onecolumn,amsmath,aps,fleqn, superscriptaddress]{revtex4}
 
\usepackage{amssymb}
\usepackage{amsmath}
\usepackage{epsfig}
\usepackage{subfigure}
\usepackage{mathrsfs}
\usepackage{longtable}
\usepackage{enumerate} 
\usepackage{multirow}
\usepackage{color}

\usepackage[usenames,dvipsnames]{xcolor}
\usepackage{hyperref}
\hypersetup{
    colorlinks = true,
    citecolor = {MidnightBlue},
    linkcolor = {BrickRed},
    urlcolor = {BrickRed}
}

\newcommand{\be}{\begin{eqnarray}}
\newcommand{\ee}{\end{eqnarray}}

\allowdisplaybreaks[1]
 
\begin{document}

\title{Covariance matrices of $\Delta \Sigma$ and $\gamma_t$ including cosmic variance}

\author{Danielle Leonard}

\maketitle
This set of notes explains how we get the {\bf statistical-only} covariance matrices required for both the Blazek et al. method and the multiple-shape-measurements method. In this document a `covariance matrix' will in general be shorthand for the covariance matrix between $r_p$ bins for the given lensing observable.
\section{$\Delta \Sigma$ (Blazek et al. method)}
In the Blazek et al. method, we need a covariance matrix for $\Delta\Sigma_a$ and for $\Delta\Sigma_b$, where $a$ and $b$ indicate the two photo-z source bins. We assume at the moment that these have totally uncorrelated covariance so we will end up combining them as covariance matrices of independent random variables.

We start with an expression for ${\rm Cov}(\gamma_t(\theta), \gamma_t(\theta^\prime))$ found in Appendix B of Jeong et al. 2009:
\begin{equation}
{\rm Cov}(\gamma_t(\theta), \gamma_t(\theta^\prime)) = \frac{1}{4\pi f_{\rm sky}} \int \frac{l \, dl}{2\pi} J_2(l \theta) J_2 (l \theta^\prime) \left[ (C_l^{g \kappa})^2 + \left(C_l^{gg} + \frac{1}{n_L}\right)\left(C_l^{\kappa \kappa} + \frac{\sigma_{\gamma}^2}{n_S}\right)\right].
\label{covgamgam}
\end{equation}
We have replaced their `halo' terms with `galaxy' terms, and $n_L$ and $n_S$ are the surface densities of, respectively, lenses and sources in number/steradian. In the case of sources this is the {\it effective} surface density value, such that $\sigma_\gamma$ is simply the rms intrinsic ellipticity. It is also the source surface density value which accounts for the weighted fraction of total galaxies in the current source bin ($a$ or $b$).

This expression uses angular separation $\theta$, whereas we are interested in an expression in terms of projected radius $r_p$, so we will immediately change to this variable using $r_p = \theta \chi_L$. It will also be helpful for us to fully multiply-out the bracketed term: 
\begin{equation}
{\rm Cov}(\gamma_t(r_p), \gamma_t(r_p^\prime)) = \frac{1}{4\pi f_{\rm sky}} \int \frac{l \, dl}{2\pi} J_2\left(l \frac{r_p}{\chi_L}\right) J_2 \left(l \frac{r_p^\prime}{\chi_L}\right) \left[ (C_l^{g \kappa})^2 + \frac{ \gamma^2 C_l^{gg}}{n_S} + \frac{C_l^{\kappa \kappa}}{n_L} + C_l^{gg} C_l^{\kappa \kappa} + \frac{\sigma_{\gamma}^2}{n_S n_L}\right].
\label{covgamgam_rp}
\end{equation}
In order to convert this to an expression for the covariance of $\Delta \Sigma$, we must introduce appropriate factors of the estimated critical surface density $\Sigma_c$ (estimated in the sense of using source photometric redshifts). In the same vein, in order to get the covariance matrices of $\Delta \Sigma$ for two samples which are defined by cuts on photometric redshifts of source galaxies, any power spectrum term which involes source galaxies (i.e. $C_l^{g\kappa}$ and $C_l^{\kappa \kappa}$) must account for these cuts. We make these adjustments to get:
\begin{align}
{\rm Cov}(\Delta \Sigma(r_p), \Delta \Sigma(r_p^\prime)) &= \frac{1}{4\pi f_{\rm sky}\bar{w}^2} \int \frac{l \, dl}{2\pi} J_2\left(l \frac{r_p}{\chi_L}\right) J_2 \left(l \frac{r_p^\prime}{\chi_L}\right) \int_{z_p^{\rm min}}^{z_p^{\rm max}} d z_p \Sigma_c^{-1}(z_p) \int_{z_p^{\rm min}}^{z_p^{\rm max}} dz_p^\prime \Sigma_c^{-1}(z_p^\prime)\nonumber \\ &\times \left[ (C_l^{g \kappa}(z_p)C_l^{g \kappa}(z_p^\prime)  + \frac{dN}{dz_p}\frac{dN}{dz_p^\prime}\frac{ \gamma^2 C_l^{gg}}{n_S} + \frac{C_l^{\kappa \kappa}(z_p, z_p^\prime)}{n_L} + C_l^{gg} C_l^{\kappa \kappa}(z_p, z_p^\prime) + \frac{dN}{dz_p}\frac{dN}{dz_p^\prime}\frac{\sigma_{\gamma}^2}{n_S n_L}\right].
\label{covDSDS_rp}
\end{align}
where spectroscopic redshift distributions and photometric redshift probability distributions are hidden inside all power spectra other than $C_l^{gg}$, and $\bar{w}$ is given by
\begin{equation}
\bar{w} = \int_{z_p^{\rm min}}^{z_p^{\rm max}} d z_p \Sigma_c^{-2}(z_p).
\label{wbar}
\end{equation}
The reason we have this factor of $\bar{w}$ and two factors of inverse $\Sigma_c$ in equation \ref{covDSDS_rp} is because in fact what is required is a weighted sum over $\Sigma_c$ - the fact that the weights go like $\Sigma_c^{-2}$ and non-z-dependent parts cancel enforces this structure. We should also note that $\frac{dN}{dz_p}$ is here normalized such that:
\begin{equation}
\int_{z_p^{\rm min}}^{z_p^{\rm max}} d z_p \frac{dN}{dz_p} = 1.
\end{equation}

In reality we don't want the covariance between specific points in $r_p$ and $r_p^\prime$, but between bins in these variables. We therefore introduce averaging over projected radial bins to get:
\begin{align}
{\rm Cov}(\Delta \Sigma(r_p^i), \Delta \Sigma(r_p^j)) &= \frac{1}{4\pi f_{\rm sky}\bar{w}^2}\frac{2}{((r^i_{\rm max})^2 - (r^i_{\rm min})^2)}\frac{2}{((r^j_{\rm max})^2 - (r^j_{\rm min})^2)} \int_{r^i_{\rm min}}^{r^i_{\rm max}} dr_p r_p \int_{r^j_{\rm min}}^{r^j_{\rm max}} dr_p^\prime r_p^\prime \nonumber \\ &\times\int \frac{l \, dl}{2\pi} J_2\left(l \frac{r_p}{\chi_L}\right) J_2 \left(l \frac{r_p^\prime}{\chi_L}\right) \int_{z_p^{\rm min}}^{z_p^{\rm max}} d z_p \Sigma_c^{-1}(z_p) \int_{z_p^{\rm min}}^{z_p^{\rm max}} dz_p^\prime \Sigma_c^{-1}(z_p^\prime)\nonumber \\ &\times \left[ (C_l^{g \kappa}(z_p)C_l^{g \kappa}(z_p^\prime)  + \frac{dN}{dz_p}\frac{dN}{dz_p^\prime}\frac{ \gamma^2 C_l^{gg}}{n_S} + \frac{C_l^{\kappa \kappa}(z_p, z_p^\prime)}{n_L} + C_l^{gg} C_l^{\kappa \kappa}(z_p, z_p^\prime) + \frac{dN}{dz_p}\frac{dN}{dz_p^\prime}\frac{\sigma_{\gamma}^2}{n_S n_L}\right].
\label{covDSDS_rpavg}
\end{align}
where recall that the averaging takes this form because $r_p$ is a polar radial coordinate. 

Consider now the final term in brackets of equation \ref{covDSDS_rpavg}. This is the term which accounts for shape noise, and does not depend on $l$. We therefore find that, considering only this term, we have:
\begin{align}
{\rm Cov}^{\rm SN}(\Delta \Sigma(r_p^i), \Delta \Sigma(r_p^j)) &= \frac{1}{4\pi f_{\rm sky}\bar{w}^2}\frac{2}{((r^i_{\rm max})^2 - (r^i_{\rm min})^2)}\frac{2}{((r^j_{\rm max})^2 - (r^j_{\rm min})^2)} \int_{r^i_{\rm min}}^{r^i_{\rm max}} dr_p r_p \int_{r^j_{\rm min}}^{r^j_{\rm max}} dr_p^\prime r_p^\prime \nonumber \\ &\times\int \frac{l \, dl}{2\pi} J_2\left(l \frac{r_p}{\chi_L}\right) J_2 \left(l \frac{r_p^\prime}{\chi_L}\right) \int_{z_p^{\rm min}}^{z_p^{\rm max}} d z_p \Sigma_c^{-1}(z_p)\frac{dN}{dz_p} \int_{z_p^{\rm min}}^{z_p^{\rm max}} dz_p^\prime \Sigma_c^{-1}(z_p^\prime) \frac{dN}{dz_p^\prime}\frac{\sigma_{\gamma}^2}{n_S n_L} \nonumber \\
&= \frac{1}{4\pi f_{\rm sky}\bar{w}^2}\frac{2}{((r^i_{\rm max})^2 - (r^i_{\rm min})^2)}\frac{2}{((r^j_{\rm max})^2 - (r^j_{\rm min})^2)} \int_{r^i_{\rm min}}^{r^i_{\rm max}} dr_p r_p \int_{r^j_{\rm min}}^{r^j_{\rm max}} dr_p^\prime r_p^\prime \nonumber \\ &\times \frac{\delta(r_p - r_p^\prime) \chi_L^2}{2 \pi r_p} \int_{z_p^{\rm min}}^{z_p^{\rm max}} d z_p \Sigma_c^{-1}(z_p)\frac{dN}{dz_p} \int_{z_p^{\rm min}}^{z_p^{\rm max}} dz_p^\prime \Sigma_c^{-1}(z_p^\prime) \frac{dN}{dz_p^\prime}\frac{\sigma_{\gamma}^2}{n_S n_L} \nonumber \\
&= \frac{\chi_L^2\delta_{ij}}{4\pi^2 f_{\rm sky}\bar{w}^2((r^i_{\rm max})^2 - (r^i_{\rm min})^2)} \int_{z_p^{\rm min}}^{z_p^{\rm max}} d z_p \Sigma_c^{-1}(z_p)\frac{dN}{dz_p} \int_{z_p^{\rm min}}^{z_p^{\rm max}} dz_p^\prime \Sigma_c^{-1}(z_p^\prime) \frac{dN}{dz_p^\prime}\frac{\sigma_{\gamma}^2}{n_S n_L}.
\label{covDSDS_rpavg_SN}
\end{align}
Because the number of sources in a bin is equal to $n_s \pi ((r^i_{\rm max})^2 - (r^i_{\rm min})^2) / \chi_L^2$ and the number lenses in equal to $n_L 4\pi f_{\rm sky}$, this is equal to the known real-space expression for the shape-noise contribution to this convariance. Since this term can be obtained analytically it is beneficial to do so, because the corresponding integrand persists to high $l$ and causes numerical instability, whereas all other terms die out at high $l$. Our final expression is therefore:
\begin{align}
{\rm Cov}(\Delta \Sigma(r_p), \Delta \Sigma(r_p^\prime)) &= \frac{1}{4\pi f_{\rm sky}\bar{w}^2}\frac{2}{((r^i_{\rm max})^2 - (r^i_{\rm min})^2)}\frac{2}{((r^j_{\rm max})^2 - (r^j_{\rm min})^2)} \int_{r^i_{\rm min}}^{r^i_{\rm max}} dr_p r_p \int_{r^j_{\rm min}}^{r^j_{\rm max}} dr_p^\prime r_p^\prime \nonumber \\ &\times\int \frac{l \, dl}{2\pi} J_2\left(l \frac{r_p}{\chi_L}\right) J_2 \left(l \frac{r_p^\prime}{\chi_L}\right) \int_{z_p^{\rm min}}^{z_p^{\rm max}} d z_p \Sigma_c^{-1}(z_p) \int_{z_p^{\rm min}}^{z_p^{\rm max}} dz_p^\prime \Sigma_c^{-1}(z_p^\prime)\nonumber \\ &\times \left[ (C_l^{g \kappa}(z_p)C_l^{g \kappa}(z_p^\prime)  + \frac{dN}{dz_p}\frac{dN}{dz_p^\prime}\frac{ \gamma^2 C_l^{gg}}{n_S} + \frac{C_l^{\kappa \kappa}(z_p, z_p^\prime)}{n_L} + C_l^{gg} C_l^{\kappa \kappa}(z_p, z_p^\prime)\right]\nonumber \\  &+ {\rm Cov}^{\rm SN}(\Delta \Sigma(r_p), \Delta \Sigma(r_p^\prime)).
\label{covDSDS_rpavg}
\end{align}

\section{$\gamma_t$ (multiple-shape-measurements method)}
For the method with multiple shape-measurement methods, we ultimately require the covariance matrix in projected radial bins of the difference of tangential shear measured by two techniques: ${\rm Cov}(\gamma_t^1(r_p) - \gamma_t^2(r_p), \gamma_t^1(r_p^\prime) - \gamma_t^2(r_p^\prime))$. The tangential shear measured from these two shape-measurement methods should be considered as correlated random variables and not assumed to be independent (contrary to the above case).

We will first consider ${\rm Cov}(\gamma_t(r_p), \gamma_t(r_p^\prime))$ and then use this to get ${\rm Cov}(\gamma_t^1(r_p) - \gamma_t^2(r_p), \gamma_t^1(r_p^\prime) - \gamma_t^2(r_p^\prime))$.

${\rm Cov}(\gamma_t(r_p), \gamma_t(r_p^\prime))$ is given by an expression very similar to equation \ref{covDSDS_rp} above. We adjust that equation to account for the fact that in the case of tangential shear no factors of critical surface density are required, and weights are redshift-independent, to get:
\begin{align}
{\rm Cov}(\gamma_t(r_p), \gamma_t(r_p^\prime)) &= \frac{1}{4\pi f_{\rm sky}} \int \frac{l \, dl}{2\pi} J_2\left(l \frac{r_p}{\chi_L}\right) J_2 \left(l \frac{r_p^\prime}{\chi_L}\right) \Bigg[ \int_{z_p^{\rm min}}^{z_p^{\rm max}} d z_p  \int_{z_p^{\rm min}}^{z_p^{\rm max}} dz_p^\prime C_l^{g \kappa}(z_p)C_l^{g \kappa}(z_p^\prime)  \nonumber \\ &\times + \frac{ \gamma^2 C_l^{gg}}{n_S} + \int_{z_p^{\rm min}}^{z_p^{\rm max}} d z_p  \int_{z_p^{\rm min}}^{z_p^{\rm max}} dz_p^\prime\frac{C_l^{\kappa \kappa}(z_p, z_p^\prime)}{n_L} + C_l^{gg} C_l^{\kappa \kappa}(z_p, z_p^\prime) + \frac{\sigma_{\gamma}^2}{n_S n_L}\Bigg].
\label{covgtgt_rp}
\end{align}
where again factors of the spectroscopic redshift distribution and photometric redshift probability distribution are hidden inside power spectra that depend on sources.

Once again, we can take advantage of Bessel function identities to perform analytically the integral in $l$ over the final shape noise term. Averaging over $r_p$ as before, we get:
\begin{align}
{\rm Cov}^{\rm SN}(\gamma_t(r_p^i), \gamma_t(r_p^j)) &= \frac{\chi_L^2 \delta_{ij}}{4\pi^2 f_{\rm sky}((r^i_{\rm max})^2 - (r^i_{\rm min})^2)}\frac{\sigma_{\gamma}^2}{n_S n_L}
\label{SNgam}
\end{align}
Once again it is sensible, in order to achieve numerical convergence, to split our final expression along these lines. We average over $r_p$ in the full expression to get:
\begin{align}
{\rm Cov}(\gamma_t(r_p^i), \gamma_t(r_p^j)) &= \frac{1}{\pi f_{\rm sky}((r^i_{\rm max})^2 - (r^i_{\rm min})^2)((r^j_{\rm max})^2 - (r^j_{\rm min})^2)}  \int_{r^i_{\rm min}}^{r^i_{\rm max}} dr_p r_p \int_{r^j_{\rm min}}^{r^j_{\rm max}} dr_p^\prime r_p^\prime \int \frac{l \, dl}{2\pi} J_2\left(l \frac{r_p}{\chi_L}\right) J_2 \left(l \frac{r_p^\prime}{\chi_L}\right) \nonumber \\ &\Bigg[ \int_{z_p^{\rm min}}^{z_p^{\rm max}} d z_p  \int_{z_p^{\rm min}}^{z_p^{\rm max}} dz_p^\prime(C_l^{g \kappa}(z_p)C_l^{g \kappa}(z_p^\prime)   + \frac{ \gamma^2 C_l^{gg}}{n_S} + \int_{z_p^{\rm min}}^{z_p^{\rm max}} d z_p  \int_{z_p^{\rm min}}^{z_p^{\rm max}} dz_p^\prime\frac{C_l^{\kappa \kappa}(z_p, z_p^\prime)}{n_L} + C_l^{gg} C_l^{\kappa \kappa}(z_p, z_p^\prime)\Bigg] \nonumber \\ &+ {\rm Cov}^{\rm SN}(\gamma_t(r_p), \gamma_t(r_p^\prime)).
\label{covgtgt_rp_final}
\end{align}

The question now becomes: how does equation \ref{covgtgt_rp_final} relate to ${\rm Cov}(\gamma_t^1(r_p) - \gamma_t^2(r_p), \gamma_t^1(r_p^\prime) - \gamma_t^2(r_p^\prime))$? Through simple algebra we can easily show:
\begin{align}
{\rm Cov}(\gamma_t^1(r_p) - \gamma_t^2(r_p), \gamma_t^1(r_p^\prime) - \gamma_t^2(r_p^\prime)) & = \langle (\gamma_t^1(r_p) - \gamma_t^2(r_p))(\gamma_t^1(r_p^\prime) - \gamma_t^2(r_p^\prime))\rangle - \langle (\gamma_t^1(r_p) - \gamma_t^2(r_p)) \rangle \langle(\gamma_t^1(r_p^\prime) - \gamma_t^2(r_p^\prime))\rangle \nonumber \\ &= {\rm Cov}(\gamma_t^1(r_p), \gamma_t^1(r_p^\prime)) + {\rm Cov}(\gamma_t^2(r_p), \gamma_t^2(r_p^\prime)) \nonumber \\ &- {\rm Cov}(\gamma_t^2(r_p), \gamma_t^1(r_p^\prime)) - {\rm Cov}(\gamma_t^1(r_p), \gamma_t^2(r_p^\prime))
\label{covexpandgam}
\end{align}
The first two terms are straightforwardly given by equation \ref{covgtgt_rp_final}. The second two terms are more tricky: they each demand the covariance of $\gamma_t$ measured using method 1 in a single projected radial bin against $\gamma_t$ measured using method 2 in another projected radial bin. This is given by:
\begin{align}
{\rm Cov}(\gamma_t^1(r_p^i), \gamma_t^2(r_p^j)) &= \frac{1}{\pi f_{\rm sky}((r^i_{\rm max})^2 - (r^i_{\rm min})^2)((r^j_{\rm max})^2 - (r^j_{\rm min})^2)}  \int_{r^i_{\rm min}}^{r^i_{\rm max}} dr_p r_p \int_{r^j_{\rm min}}^{r^j_{\rm max}} dr_p^\prime r_p^\prime \int \frac{l \, dl}{2\pi} J_2\left(l \frac{r_p}{\chi_L}\right) J_2 \left(l \frac{r_p^\prime}{\chi_L}\right) \nonumber \\ &\Bigg[ \int_{z_p^{\rm min}}^{z_p^{\rm max}} d z_p  \int_{z_p^{\rm min}}^{z_p^{\rm max}} dz_p^\prime C_l^{g \kappa}(z_p)C_l^{g \kappa}(z_p^\prime) + \int_{z_p^{\rm min}}^{z_p^{\rm max}} d z_p  \int_{z_p^{\rm min}}^{z_p^{\rm max}} dz_p^\prime\frac{C_l^{\kappa \kappa}(z_p, z_p^\prime)}{n_L} \nonumber \\ &+ C_l^{gg} C_l^{\kappa \kappa}(z_p, z_p^\prime)  + {\rm Corr}_{\gamma^2}\left(\frac{C_l^{gg}}{n_S} + \frac{1}{n_S n_L} \right)  \Bigg]
\label{covgtgt_diffmeth}
\end{align}
This expression is explained as follows: the power-spectra terms source both variance when $i=j$ and covariance with $i \ne j$, and are present regardless of the method of shape-measurement. ${\rm Corr}_{\gamma^2}$ quantifies the correlation in the rms shape given by by each methods. It is given more usefully by:
\begin{equation}
{\rm Corr}_{\gamma^2} = \rho_{\gamma^2} \sigma_{\gamma}^{1}\sigma_{\gamma}^{2}
\label{corr_gam}
\end{equation}
where $\rho_{\gamma^2}$ is the pearson correlation coefficient between $\sigma_{\gamma}^{1} $ and $\sigma_{\gamma}^{2}$. This is an input variable to our calculation. Note that in equation \ref{covgtgt_diffmeth} we have left it implicit that in the case $i \ne j$ the constant term proportional to ${\rm Corr}_{\gamma^2}$ is equal to 0, we will account for this in our final expression.

Plugging in now to equation \ref{covexpandgam}, we find that the covariance of differences we need is given by:
\begin{align}
{\rm Cov}(\gamma_t^1(r_p^i) - \gamma_t^2(r_p^i), \gamma_t^1(r_p^j) - \gamma_t^2(r_p^j))&= \frac{1}{\pi f_{\rm sky}((r^i_{\rm max})^2 - (r^i_{\rm min})^2)((r^j_{\rm max})^2 - (r^j_{\rm min})^2)}  \int_{r^i_{\rm min}}^{r^i_{\rm max}} dr_p r_p \int_{r^j_{\rm min}}^{r^j_{\rm max}} dr_p^\prime r_p^\prime \nonumber \\ &\times \int \frac{l \, dl}{2\pi} J_2\left(l \frac{r_p}{\chi_L}\right) J_2 \left(l \frac{r_p^\prime}{\chi_L}\right) \nonumber \\ &\times \Bigg[(\sigma_{\gamma}^{1})^2\left(\frac{C_l^{gg}}{n_s} + \frac{1}{n_L n_s}\right) + (\sigma_{\gamma}^{2})^2\left(\frac{C_l^{gg}}{n_s} + \frac{1}{n_L n_s}\right)- 2\rho_{\gamma^2} \sigma_{\gamma}^{1}\sigma_{\gamma}^{2} \left(\frac{C_l^{gg}}{n_S} + \frac{1}{n_S n_L} \right)  \Bigg]
\label{cov_diff}
\end{align}
and to be explicit about the fact that shape-noise-related terms only contribute to diagonal elements:
\begin{align}
{\rm Cov}(\gamma_t^1(r_p^i) - \gamma_t^2(r_p^i), \gamma_t^1(r_p^j) - \gamma_t^2(r_p^j))&= \frac{1}{\pi f_{\rm sky}((r^i_{\rm max})^2 - (r^i_{\rm min})^2)((r^j_{\rm max})^2 - (r^j_{\rm min})^2)}  \int_{r^i_{\rm min}}^{r^i_{\rm max}} dr_p r_p \int_{r^j_{\rm min}}^{r^j_{\rm max}} dr_p^\prime r_p^\prime\nonumber \\ & \int \frac{l \, dl}{2\pi} J_2\left(l \frac{r_p}{\chi_L}\right) J_2 \left(l \frac{r_p^\prime}{\chi_L}\right) \Bigg[(\sigma_{\gamma}^{1})^2 \frac{C_l^{gg}}{n_s} + (\sigma_{\gamma}^{2})^2\frac{C_l^{gg}}{n_s} - 2\rho_{\gamma^2} \sigma_{\gamma}^{1}\sigma_{\gamma}^{2}\frac{C_l^{gg}}{n_S}  \Bigg]  \nonumber \\ &+\left(\frac{\sigma_{\gamma^1}^2}{n_S n_L} +\frac{\sigma_{\gamma^2}^2}{n_S n_L} - 2 \frac{\rho_{\gamma^2} \sigma_{\gamma}^{1}\sigma_{\gamma}^{2}}{n_Sn_L}\right)\frac{\chi_L^2}{4\pi^2 f_{\rm sky}((r^i_{\rm max})^2 - (r^i_{\rm min})^2)}\delta_{ij}
\label{cov_diff_exp}
\end{align}



%-------------------------------------------------------------------------------


%-------------------------------------------------------------------------------

\end{document}
